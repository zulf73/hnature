\documentclass{article}
\title{Stable Mating Criteria for Human Females}
\author{Zulfikar Moinuddin Ahmed}
\begin{document}
\maketitle
The mating criteria used by unwed females in America between 1930-1990s were examined by Buss et. al. (2001). The ranking from most to least preferred is the following:
\begin{itemize}
\item{Emotional Stability}
\item{Dependable Character}
\item{Mutual Attraction}
\item{Pleasing disposition}
\item{Ambition, industriousness}
\end{itemize}

This seems at first not particularly interesting, but it is strongly at odds with the prevailing evolutionary theories of mate selection which emphasize physical attractiveness of females and social status and financial prospects and status for males.  These do belong to the list, but much higher than rank 10 for both males and females.

\section{Association of Criteria to Big Five Model}

The Big Five Model of McRae and Costa and others came from a different tradition of descriptive personality traits.  The mnemonic is OCEAN, with O=Openness, C=Conscientiousness, E=Extraversion,
A=Agreeableness, N=Neurotocism.  We can now take the measured data on female mate preferences and make the natural association of (1) Emotional Stability to (negative) Neuroticism, (2) Dependable Character to Conscientiousness, (4) Pleasing disposition with Agreeableness.  This association allows us to immeditely merge two strains of research, the descriptive personality traits on one side, and on sexual mating psychology of humans on the other side in a {\em natural manner}.  

The naturality here is that the Buss 2001 paper was not interested in personality traits but actual characteristics sought in mates by females.  It is in this sense not contrived data.  

It is purely serendipitious discovery of mine that the top five ranked desired characteristics have natural association to the big five personality trait model.  This is a tremendously fortunate discovery because there is a great deal of confusion still in our collective understanding of two separate questions.

One the one hand a natural question is what is the evolutionary explanation for effects on the big five personality traits.  On the other hand the question is what are the personality traits involved in the explicit desires of females for mates.  Conventional, and misguided answers to the latter had been that the main issues are sexual attractiveness and youth of females and social status and resources of the males are the major priority.  In fact, neither of these appear in the top five desires.  Instead, we find that personality traits of Neuroticism, Agreeableness, and Conscientiousness are directly demanded by both females and males for desirable characteristics of mates.  

\section{Simple interpretation}

The simple interpretation is that human females and males have instinctively developed an intuition to avoid mates who will be predatory, aggressive, and harmful in various ways to their interests and instead be friendly and open.  Thus here it is not eros but rather thanatos that is driving the top choice for mating -- i.e. emotional stability -- in the simplest view is to avoid threats of murder if there is high neuroticism.  It is known that neuroticism has the component of the fear and aggressive reactions as well as other malaises that threaten the female.  It is interesting to see that these thanatos considerations also are in the male mate choices as top priorities too.  

This may seem simple, but because the data come from measurements rather than speculation that emotional stability is indeed the top choice.

Let us look at the data and ranking by average of 1930s to 1990s of American women's mate preferences.

% latex table generated in R 3.6.3 by xtable 1.8-4 package
% Fri Oct  9 00:31:19 2020
\begin{table}[ht]
\centering
\begin{tabular}{rllrrrrrrll}
  \hline
 & V1 & V2 & V3 & V4 & V5 & V6 & V7 & V8 & V9 & V10 \\ 
  \hline
1 & Rank & Period &  30 &  50 &  60 &  70 &  80 &  90 & Avg & Surprise \\ 
  2 & 1 & Emotional stability &   1 &   2 &   1 &   2 &   2 &   3 & 1.833333333 & Neuroticism \\ 
  3 & 2 & Dependable character &   2 &   1 &   2 &   3 &   3 &   2 & 2.166666667 & Moral Virtues \\ 
  4 & 3 & Mutual Attraction &   5 &   6 &   3 &   1 &   1 &   1 & 2.833333333 & Romantic Love \\ 
  5 & 4 & Pleasing disposition &   4 &   5 &   4 &   4 &   4 &   4 & 4.166666667 & Agreeableness \\ 
  6 & 5 & Ambition, industrious &   3 &   4 &   6 &   6 &   6 &   7 & 5.333333333 & Character  \\ 
  7 & 6 & Desire for children &   7 &   3 &   5 &  10 &   7 &   6 & 6.333333333 & Family \\ 
  8 & 7 & Education, intelligence &   9 &  14 &   7 &   5 &   5 &   5 & 7.5 & Exp higher rank \\ 
  9 & 8 & Good health &   6 &   9 &  10 &   8 &   9 &   9 & 8.5 & Health \\ 
  10 & 9 & Similar Educational Background &  12 &   8 &   9 &   9 &  10 &  10 & 9.666666667 & 30 percent college \\ 
  11 & 10 & Sociability &  11 &  11 &  13 &   7 &   8 &   8 & 9.666666667 &  \\ 
  12 & 11 & Refinement, neatness &   8 &   7 &   8 &  12 &  12 &  12 & 9.833333333 & Aristocracy \\ 
  13 & 12 & Good financial prospect &  13 &  12 &  12 &  11 &  11 &  11 & 11.66666667 & Exp higher rank \\ 
  14 & 13 & Similar religious background &  14 &  10 &  11 &  13 &  15 &  14 & 12.83333333 &  \\ 
  15 & 14 & Favorable social status &  15 &  13 &  14 &  14 &  14 &  15 & 14.16666667 & Exp higher rank \\ 
  16 & 15 & Good Looks &  17 &  18 &  17 &  15 &  13 &  13 & 15.5 & Exp higher rank \\ 
  17 & 16 & Chastity &  10 &  15 &  15 &  18 &  18 &  17 & 15.5 & Virtue \\ 
  18 & 17 & Similar political background &  18 &  17 &  18 &  17 &  17 &  18 & 17.5 &  \\ 
   \hline
\end{tabular}
\end{table}

The equivalent for males is the following.

% latex table generated in R 3.6.3 by xtable 1.8-4 package
% Fri Oct  9 00:33:01 2020
\begin{table}[ht]
\centering
\begin{tabular}{rllrrrrrrll}
  \hline
 & V1 & V2 & V3 & V4 & V5 & V6 & V7 & V8 & V9 & V10 \\ 
  \hline
1 & Rank & Period &  30 &  50 &  60 &  70 &  80 &  90 & Avg & Surprise \\ 
  2 & 1 & Emotional stability &   1 &   2 &   1 &   2 &   2 &   3 & 1.833333333 & Neuroticism \\ 
  3 & 2 & Dependable character &   2 &   1 &   2 &   3 &   3 &   2 & 2.166666667 & Moral Virtues \\ 
  4 & 3 & Mutual Attraction &   5 &   6 &   3 &   1 &   1 &   1 & 2.833333333 & Romantic Love \\ 
  5 & 4 & Pleasing disposition &   4 &   5 &   4 &   4 &   4 &   4 & 4.166666667 & Agreeableness \\ 
  6 & 5 & Ambition, industrious &   3 &   4 &   6 &   6 &   6 &   7 & 5.333333333 & Character  \\ 
  7 & 6 & Desire for children &   7 &   3 &   5 &  10 &   7 &   6 & 6.333333333 & Family \\ 
  8 & 7 & Education, intelligence &   9 &  14 &   7 &   5 &   5 &   5 & 7.5 & Exp higher rank \\ 
  9 & 8 & Good health &   6 &   9 &  10 &   8 &   9 &   9 & 8.5 & Health \\ 
  10 & 9 & Similar Educational Background &  12 &   8 &   9 &   9 &  10 &  10 & 9.666666667 & 30 percent college \\ 
  11 & 10 & Sociability &  11 &  11 &  13 &   7 &   8 &   8 & 9.666666667 &  \\ 
  12 & 11 & Refinement, neatness &   8 &   7 &   8 &  12 &  12 &  12 & 9.833333333 & Aristocracy \\ 
  13 & 12 & Good financial prospect &  13 &  12 &  12 &  11 &  11 &  11 & 11.66666667 & Exp higher rank \\ 
  14 & 13 & Similar religious background &  14 &  10 &  11 &  13 &  15 &  14 & 12.83333333 &  \\ 
  15 & 14 & Favorable social status &  15 &  13 &  14 &  14 &  14 &  15 & 14.16666667 & Exp higher rank \\ 
  16 & 15 & Good Looks &  17 &  18 &  17 &  15 &  13 &  13 & 15.5 & Exp higher rank \\ 
  17 & 16 & Chastity &  10 &  15 &  15 &  18 &  18 &  17 & 15.5 & Virtue \\ 
  18 & 17 & Similar political background &  18 &  17 &  18 &  17 &  17 &  18 & 17.5 &  \\ 
   \hline
\end{tabular}
\end{table}

\section{Paradigm Shift}
Scientific theories are only valuable if they match observations; otherwise they need to be considered provisional.  Here we can just look at direct observation of mate characteristics and recognize that the features targeted for modeling of human reproduction such as sexual attractiveness and social and financial prospects are well below rank 10 out of 17 and therefore are not primary driving forces determining coupling and reproduction in humans.  Therefore all correct theories must necessarily reproduce this ranking.  This ranking is as close to ground truth that one can find in the evolutionary psychology.  This will allow us to produce sharper and more accurate theories in the future.

\begin{thebibliography}
 bibitem David Buss et. al. Half a Century of Mate Preferences: Changes in Values, J. Family and Marriage, 63, 2001, 491--503
\end{thebibliography}
\end{document}
