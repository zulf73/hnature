\documentclass{amsart}
\usepackage{graphicx}
\graphicspath{{./}}
\usepackage{hyperref}
\usepackage{csvsimple}
\usepackage{longtable}
\usepackage{epigraph}
\title{Zulf's Vision for a Science of Quantitative Human Nature}
\author{Zulfikar Moinuddin Ahmed}
\date{\today}
\begin{document}
\maketitle

\section{Background}

Mathematical analysis developed with quantum theory in the twentieth century.  David Hilbert was completely mesmerised by the work of Erik Ivar Fredholm of 1900 on integral equations and developed Hilbert space theory, and then many mathematicians developed the theory of unbounded self-adjoint operators with a focus on quantum mechanical observables as operators.  Now my personal view is clear enough, that the Hilbert spaces that matter in actual physics are $L^2(\Gamma(\Sigma S^4))$ the smooth sections of spinor fields of a four-sphere.  The general infrastructure of quantum mechanics is from my point of view not as valuable in physics.

But what all this work of mathematics is valuable for us today is to understand the mathematical infrastructure needed for Human Nature.  You see, here, we don't want a very general abstract situation.  We want to consider Human Nature metrics to be analogues of the sort that occur in World Values Survey.  We expect the high dimensional dataset $Y=(Y_1,\dots, Y_{430})$ to produce a (discrete) correlation matrix {\em whose spectrum is an approximation of a Generalised Hyperbolic Distribution}.  Take this on faith for now and I will return with some actual fits later.

So what is this?  It is the {\em spectrum of a self-adjoint operator on an infinite dimensional Hilbert space}.  So we need to take this as a serious mathematical situation and carefully understand how to produce relevant mathematics for it.  It is likely not a major change in point of view from Hilbert spaces in quantum theory but I am not used to operators whose spectrum is a smooth density with mass concentrated in a compact region of $\mathbf{R}$. 

Now this may not be a mathematically challenging path at all; or it might be.  Regardless, this development is necessary for Quantitative Human Nature, and it is a great project in deeper and clearer understanding of mathematics too.  

You see, I am used to discrete spectra in mathematical problems.  These continuous spectra were not interesting before at all.  So I think this might require new clear mathematical understanding as well.

\section{Continuous Spectra Are Demanding}

We are in a new situation for me.  I do not doubt that there are many great people with knowledge and experience with operators with spectra that behave like Gaussians but I don't have this experience, and I consider this to be crucial to making progress in Quantitative Human Nature as a Science.

\section{General Appeal To Mathematicians}

Mathematicians ought to work out a theory of self-adjoint operators on Hilbert spaces whose spectrum density -- appropriately defined -- is a Generalised Hyperbolic Distribution.  This is what Social Science will need in the future to have sharp understanding of Quantitative Human Nature.  

Mathematicians ought to be very clear about what the issues are and present it to Social Scientists with minimal fuss and bother, and this will lead to a vast advance in Social Science and produce much finer tools to understand Human Nature and Human Behavior in a coherent mathematical context.  The subtleties are unknown.

This is a mutually beneficial affair.  On one hand Mathematicians are lost without hard resistance from Nature, and deepest mathematical questions are inevitably questions that are important to understand Nature's mysteries.  On the other hand Social Scientists are today in a dreadful {\em pre-scientific} era using inadequate tools such as low correlations without significance which cannot be understood without a continuous context and statistical techniques that are not appropriate for deep and subtle issues that are hidden altogether without an appropriate context.  Appropriate tools come from deeper mathematical understanding and here there is an infinite dimensional context that cannot be well-approximated by ad hoc tools of computer and data scientists who are blind in their approach and over-optimistic about how much precision can result from Artificial Intelligence prediction schemes.

Be good Angels, Mathematicians, and do this, and help out your brothers who will nourish you with great and deep beauty in Human Nature.


\end{document}