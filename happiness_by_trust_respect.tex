\documentclass{amsart} 
\usepackage{graphicx}
\graphicspath{{./}}
\usepackage[fontsize=14pt]{scrextend}
\usepackage{hyperref}
\usepackage{csvsimple}
\usepackage{epigraph}
\title{Trusting everyone and Teaching Kids Respect and Tolerance explains 33\% of Happiness Variation}
\author{Zulfikar Moinuddin Ahmed}
\date{\today}
\begin{document}
\maketitle

\begin{verbatim}
> summary(mod.haptrust)

Call:
lm(formula = trustMost ~ Happy + childResp, data = haptrust)

Residuals:
    Min      1Q  Median      3Q     Max 
-29.996 -11.533   0.071  11.567  38.429 

Coefficients:
            Estimate Std. Error t value Pr(>|t|)    
(Intercept) -79.5511    18.9127  -4.206 7.03e-05 ***
Happy         0.8173     0.2090   3.910 0.000199 ***
childResp     0.5367     0.1254   4.280 5.40e-05 ***
---
Signif. codes:  0 ‘***’ 0.001 ‘**’ 0.01 ‘*’ 0.05 ‘.’ 0.1 ‘ ’ 1

Residual standard error: 14.93 on 76 degrees of freedom
Multiple R-squared:  0.3294,	Adjusted R-squared:  0.3117 
F-statistic: 18.66 on 2 and 76 DF,  p-value: 2.551e-07
\end{verbatim}

\section{Teaching Kids Respect is Proxy for being Respected oneself}

Tay and Diener decomposed SWB into five factors.  Beyond Basic needs, they had Social/Respect and Autonomy/Respect.  We took teaching kids tolerance and respect as a proxy.  So with two factors we can explain 33\% of the variation of happiness with $N=79$ countries.  Both variables are very strongly significant.

What is really interesting is that correlation between trust and teaching respect to kids is not zero but $\rho=0.44$ so these are not orthogonal variables.  It is interesting to consider trust as a separate variable but in the Tay-Diener scheme it would fall under "Social Needs".  Trust is obviously much more specific than general social needs.  

\end{document}
