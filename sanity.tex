\documentclass{amsart}
\usepackage{graphicx}
\graphicspath{{./}}
\usepackage{hyperref}
\usepackage{csvsimple}
\usepackage{longtable}
\usepackage{lscape}
\usepackage{epigraph}
\usepackage{amsmath}
\DeclareMathOperator*{\argmax}{arg\,max}
\DeclareMathOperator*{\argmin}{arg\,min}\title{Relatively Simple Things to Bring Sanity to My Life}
\author{Zulfikar Moinuddin Ahmed}
\date{\today}
\begin{document}
\maketitle

\section{Discrete Time Markov Chains with State Space $\mathbf{R}$}

Suppose $X_1,X_2,\dots$ is a Markov Chain with values in $\mathbf{R}$.  We let time go by, and let's assume that over time, the distribution of $X_r$, which is a probability distribution on $\mathbf{R}$ converges.  We want to understand simple things, so we don't worry about the conditions that guarantee convergence, etc.  Let's denote by $D(X_r)$ to be the distribution of $X_r$ and let's assume that 
\[
\lim_{r\rightarrow\infty } D(X_r) = \delta(x)
\]
This is our notation for many things going on here.  We assume that the distribution converges to a probability measure on $\mathbf{R}$ which happens to be absolutely continuous with respect to Lebesgue measure, and the density is $\delta(x)$.  This does not always happen and we don't care about all the possible problems where there are failures, and assume that $X_r$ satisfies various conditions so there is no problem in convergence of distributions.

\section{Map to the Circle}

Now we consider $s:\mathbf{R}\rightarrow S^1$ by
\[
s(x) = e^{2 \pi i x}
\]

What is the distribution of the $s(X_1),\dots$?

We consider the following function:
\[
\rho(t) = \sum_{k= -\infty} ^ {\infty} \delta( t + 2\pi k)
\]
The infinite series converges absolutely since $\delta(x)$ is a probability density.

\section{Lesson From the Above}

The above result is a general simple issue, that if you have a stochastic process on the state space $\mathbf{R}$ then pushing it forward under $s(x)$ leads to taking all the mass accumulated in the $(2\pi k, 2\pi (k+1)]$ intervals and then summing them up.  We are then left with a density on the circle.

\section{Is the Only Possible Invariant Measure on $S^2$ the Lebesgue Measure?}

I do not know the answer to this, so I will just do this as an exercise.  I am interested in measures on $S^2$ that arise from geometric Markov chains that move around in $S^2$.  I don't know what invariant measures in $S^2$ can arise in this way.  There are many sorts of theorems that tell us that invariant measures on compact spaces like this are restricted to just the Lebesgue measure.  

Suppose we have the stereographic projection $\tau:S^2-{\infty} \rightarrow
\mathbf{C}$.  Suppose $\mu_0$ is the rotation-invariant Lebesgue measure on $S^2$,  then $\tau_*(\mu_0)$ can be calculated explicitly.  We want to know if this is related closely to the Barndorff-Nielsen density on $\mathbf{R}$.


\end{document}