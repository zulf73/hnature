\documentclass{amsart} 
\usepackage{graphicx}
\graphicspath{{./}}
\usepackage[fontsize=14pt]{scrextend}
\usepackage{hyperref}
\title{Transcendent paths of grief, sorrow and loss}
\author{Zulfikar Moinuddin Ahmed}
\date{\today}

\begin{document}
\maketitle

\section{Introduction}

Duino Elegies repeatedly points out the transcendent potential for grief and loss as part of condition of human living.  In Pakistan, Syed Ahmad Shah (1931--2008) had developed the gazal which had developed the transcendent potential of love and loss from Persian poetic tradition.  In this brief note we will juxtapose this theme to understand the substantial content for understanding of ordinary human beings who are not devoted to poetry or spirituality of this sort.


\section{Brief biography of Ahmed Faraz}

Syed Ahmad Shah (1931-2008), who acquired the pen name of Faraz, came to be known as Ahmad Faraz as a poet. He was born at Nav Shahra although his ancestral place was Kohat. He studied at Islamia College, Kohat; Edwards College, Peshawar; and Peshawar University from where he got his degrees of M. A. in Urdu and M. A. in Persian. He began his career as a producer in Radio Pakistan. Later, he worked as a lecturer at Islamia College in Peshawar. Faraz disapproved of the military dictatorship in Pakistan and expressed himself unreservedly for which he was arrested. On his release, he preferred to live in a self-imposed exile in Europe and Canada for six years. Back home, he took up senior positions of administrative nature as Resident Director of Pakistan National Centre, and subsequently the Director of Akademy Adabiyat Pakistan, Lok Wirsa and Chairperson of National Book Foundation. A widely respected poet, Faraz received several awards. Some of these include Adamji Award, Abaseen Award, Kamal-e-Fun Award, and Hilal-e-Imtiyaz award, which he returned registering his displeasure with the country’s governance. He was decorated with Hilal-e-Pakistan award by the Government of Pakistan posthumously.  

Faraz started writing poetry while he was still a young college student. He emerged as a ghazal poet with an individual signature of his own. Even while he drew upon the traditional subjects of love and romance, he also wrote his age in his poetry with all its despairs and disappointments and produced some of the finest specimens of resistance poetry.  He was a prolific poet with several anthologies to his credit. These include Tanha Tanha, Dard-e-Aashob, Janan Janan, Shubkhoon, Merey Khwab Reza Reza, Beaawara Gali Koochon Mein, Nabeena Shar Mein Aaeena, Pus Andaz Mausam,  and Khwab-e-Gul Pareshan Hai. His translations of poetry are included in Sub Awazein Meri Hain. He also put together a selection from the poetry of Kunwar Mahinder Singh Bedi in Ai Ishq Junoon Pesha. His Kulliyat appeared with an inclusive title of Shahr-e-Sukhan Aaraasta Hai.

\section{Rilke's Tenth Elegy}

"How dear you will seem to me then, you nights of anguish.  Why didn't I kneel more deeply to accept you, inconsolable sisters, and surrendering, lose myself to your loosened hair.  How we squander our hours of pain.  How we gaze beyond them into the bitter duration to see if they have an end.  Though they are really our winter enduring foliage, our dark evergreen, one season in our inner year--not only a season in time--, but are place and settlement, foundation and soil and home."

\section{Ahmed Faraz "Ab ke hum bichde to shayad"}

This time if we part, then we may meet in the dreams

Like dry flowers that are found in books

Neither you are God nor is my love like an angel

If we both are human beings, then why should we meet in hiding

Now that we are parting our ways, may be we will meet in dreams

Why not add the sorrows of the world to the sorrows caused by my beloved

After all when the drinks are mixed, the intoxication increases

Now that we are parting our ways, may be we will meet in dreams

You and I are not the same, nor do we have the same destination Faraz

It is like two shadows meeting in the desert of wishes

Now that we are parting our ways, may be we will meet in dreams

Like dry flowers that are found in books

\section{Attitude towards sorrow}

From Ahmed Faraz, we find this extension of sorrows from lost love:

"Why not add the sorrows of the world to the sorrows caused by my beloved

After all when the drinks are mixed, the intoxication increases"

In the tradition of Urdu ghazals, the transcendent potential of sorrows of the world being pooled with sorrows from lost love as an intoxicating mix can be seen to be analogy to Rilke's interest in nights of anguish and pain  being far from a state to avoid than one to consider "not only a season in time--, but are place and settlement, foundation and soil and home"  

Ahmed Faraz' poem's suggestion has deep resonance in Eastern spirituality where Buddha from further East as well as Rama from Indus Valley history had introduced concerns for the 'sorrows of the world' as a spiritual quest while the romantic love tradition had been concerned with the anguish and pain of parting of lover and beloved as a central concern.  

\section{Substantial rather than word games}

Certainly not the tradition of gazals whose craft developed as vocal musical tradition.  


\begin{thebibliography}{CCCC}
\bibitem{MH08}{Mehdi Hassan, \url{https://www.youtube.com/watch?v=SYH4-bKsEGE}}
\end{thebibliography}
\end{document}