\documentclass{amsart} 
\usepackage{graphicx}
\graphicspath{{./}}
\usepackage[fontsize=14pt]{scrextend}
\usepackage{hyperref}
\usepackage{csvsimple}
\usepackage{epigraph}
\title{Robbery, Murder and Making Parents Proud as Goal in 18 Countries}
\author{Zulfikar Moinuddin Ahmed}
\date{\today}
\begin{document}
\maketitle
We present two linear univariate predictions of Robbery and Murder Rates using proportion of people in a country who do {\em not} have making their parents proud as a major goal.

\section{Murder is Predicted with close to 95 percent significance}

\begin{verbatim}
> summary(mod.murd.npp)

Call:
lm(formula = Murder ~ NoPProud, data = crms)

Residuals:
    Min      1Q  Median      3Q     Max 
-3.0158 -0.8879 -0.4276  0.1988  4.9189 

Coefficients:
            Estimate Std. Error t value Pr(>|t|)   
(Intercept)  4.54813    1.20666   3.769  0.00168 **
NoPProud    -0.10404    0.04993  -2.084  0.05359 . 
---
Signif. codes:  0 ‘***’ 0.001 ‘**’ 0.01 ‘*’ 0.05 ‘.’ 0.1 ‘ ’ 1

Residual standard error: 1.916 on 16 degrees of freedom
Multiple R-squared:  0.2134,	Adjusted R-squared:  0.1643 
F-statistic: 4.341 on 1 and 16 DF,  p-value: 0.05359
\end{verbatim}

The coefficient is negative for NoPProud to predict Murder Rate.

\section{Weaker result for Robbery}

\begin{verbatim}

> summary(mod.rob.npp)

Call:
lm(formula = Robbery ~ NoPProud, data = crms)

Residuals:
    Min      1Q  Median      3Q     Max 
-90.217 -49.406   0.026  28.945 115.905 

Coefficients:
            Estimate Std. Error t value Pr(>|t|)
(Intercept)   18.109     38.388   0.472    0.643
NoPProud       2.368      1.589   1.490    0.156

Residual standard error: 60.96 on 16 degrees of freedom
Multiple R-squared:  0.1219,	Adjusted R-squared:  0.06703 
F-statistic: 2.221 on 1 and 16 DF,  p-value: 0.1556
\end{verbatim}

This I do not have any explanation for: the coefficient is positive but the p-value is 0.156 for NoPProud variable.


\end{document}
