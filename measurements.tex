\documentclass{amsart}
\usepackage{graphicx}
\graphicspath{{./}}
\usepackage{hyperref}
\usepackage{csvsimple}
\usepackage{longtable}
\usepackage{lscape}
\usepackage{epigraph}
\title{Thoughts about measurements and statistical philosophy in Science}
\author{Zulfikar Moinuddin Ahmed}
\date{\today}
\begin{document}
\maketitle

\section{Measurements and Statistical Models are not Enough}

I am not in the end an empiricist from the philosophical perspective.  Even in high school, and at Princeton 1991-1995 I was a Platonist.  In fact, at Princeton I did not get deeply involved in messy world of statistical analysis of measurements.  I hid in pure mathematics and literature.  Years later this Platonism bore fruit as I challenged QED and Relativity and Expansion and Big Bang successfully.  But that was decades later starting 2008.  Four-Sphere Theory was my first truly spectacular scientific success.  I had some before but nothing of this grand scale. It was a success of extremely sharp focus on Platonic views and care in theoretical concerns.  

This success gave me a different perspective regarding empiricism and dealing with measurements and data for scientific models.  First, the great confidence I gained from seeing pure mathematical beauty in the deep features of macroscopic (physical) Nature was so astounding to me I could not breathe and was overtaken by a euphoria that cannot be described in words.  Even 10-12 decimal figure accuracy models such as Special Relativity and Quantum Electrodynamics models fell to it.  At the other hand, after 2018 especially, it dawned on me the vast abundance of ambition in human sciences, which have nothing like the strong foundations of physics.  Physics ideas refined over three centuries before being clear with Four-Sphere Theory.  And it was enormously difficult to come to the right ideas.  The difficulty is in going against vast consensus.  It's extraordinary energy that overcomes these.  

In the case of human sciences, without strong foundations that are certain, it is infinitely harder than physics to be certain what is true and what principles are correct.  

\section{Placing Empiricism in Its Proper Place}

I am not impressed by hard core empiricists who are material-oriented any more.  The universe does not actually have three spatial dimensions but four, and the extra dimension is difficult for us for it is not part of our ordinary senses. At the same time, empiricism is necessary to have any truth of nature.  

I have learned over time to have extreme reverence for Nature.  Nature is always inscrutible and does not respect our concepts about her.  She is fickle and her moments of extraordinary generosity are rare.  Our human minds are not able to even conceive of what concepts could have value for nature, let alone be able to consider any Oracle of truth.  Empiricism is a rather shabby approach to attempting to elucidate the deep and complex and inscrutible of Nature's secrets. But it is often all we have, and we remain like children with screwdrivers on Saturday finding a Lamborghini in our garage quite confident that if we use the screwdriver judiciously to pull apart the car, the nature of the Lamborghini will be laid bare.  Empiricism is that screwdriver.

\section{Truth of Single Human Race is Nontrivial Scientific Fact}

I grew up in multi-ethnic environments from 13 at least since I moved in 1987 to New York, and then attended Princeton 1991-1995 where people from many ethnicities studied.  So never did I develop any rubbish theory of intelligence based on ethnicity or other racist theories.  I habitutated myself to dealing with people as individuals.  I have had multiple white lovers and never considered white people or anyone else a different 'race' than myself.  Actually I thought all that was from before the Second World War in creepy Nazi films or from the Civil Rights movement of the 1960s, before I was born.  This then allowed me to become quite curious about racism.  I dealt with some racism in random places in America, Greyhound buses, Williamsburg Diners, etc.  But I was too internal to really worry about it.  Then I came across some on Facebook, racial types.  I was not impressed overall with their intelligence.  They're idiots most of these race oriented people and then I dug into the scientific evidence for single human race finally, because it seemed to be quite obvious to me but seemed not to be well-known.

It became clear to me that I would have to do some actual work to understand ethnicity and I did some of this work with moral values, preferences, and childrearing values.  The genetic closeness of ethnicities is not as compelling for social acceptance.  My work on ethnicity effects on moral values for human race and preferences is the scientific evidence that matters because it is in moral and other qualities that people actually judge character.  The uniformity is surprising but is quite deep truth about the human race.  These are interesting partly because I myself simply did not know, and also because Bill Gates was such a horrible malevolent cunt who is a white supremacist and I felt there was an opportunity to change the world with some scientific models here.

\section{Quick Review of History of Statistics}

I have been quite focused on understanding Human Nature through the beautiful high quality data of the World Values Surveys with a large number of questions and roughly 70,000 respondents.  The Single Human Race and Universal Human Nature have been on my mind for many months now.  I just looked over Stigler's book on History of Statistics.  I will have a sharper sense of the history at some points.  But I established some themes for myself.  Combinations of observations in astronomy and geodesy on one hand and inference from probability models first merge in 1810.  Closer to nineteenth century, Francis Galton, Edgeworth, Pearson and Yule have success in social sciences and establish the field of statistics.  I am doing Social Science when working with World Values Survey data.  And suddenly it dawns on me that it is not an accident that I had to invent the discrete correlation of categorical variables rejecting the Pearson chi-square independence test. This short history suddenly brings to fore that it is not a surprise that I am worried about fundamental issues of statistical inference.  In an uncanny way, I am facing the same problems as Galton, Edgeworth, Pearson, and Yule, with new perspectives.  I would like a Quantitative Social Science, and I am not satisfied with the tools of these pioneers.  I have a new valid discrete correlation (since Pearson's was not good) and a new chi-square test.  And Barndorff-Nielsen's Generalised Hyperbolic Distribution is absolutely central for the future of Social Science.  My work thus far not only showed the power of the high quality global data to make strong inferences that can transform the understanding of unity of Human Race but also review the scientific infrastructure of Social Science from its foundations, because the tools for Social Science are precisely the basic tools of statistics.

\section{Certain Truth and Uncertainties of Nature Review}

The Scientific models of Nature are mathematical models by and large, and the issues of whether a model will fit Nature is a matter of statistical inference from data.  It was betwee 1641 when Rene Descartes wrote his {\em Meditations on the First Philosophy} and roughly the birth of science proper in Europe to around 1730s when the issues of how to gain knowledge of {\em true values} of varibles became clear.  It was Adrien Marie Legendre who published on the least-squares method in 1805.  Now he was a mathematician and wrote just about the fitting of coefficients based on $L^2$ error.  A separate stream in probability from games of chance starting with computations of games of chance to De Moivre's computations for sample size for the number of samples before the mean is reliable to some specified accuracy was examined.  There was an optimism of these pioneers that the values with a specified precision is a valid statistical inference assuming Gaussian noise model.  Their terminology was different but this is what developed in fact, that the pioneers such as De Moivre were interested in gaining certain knowledge about quantities in nature and were interested in the number of measurements needed before certain knowledge can be asserted for the value of a variable.  

This is a remarkable issue that is actually quite close to my heart.  You might be surprised to learn that for a {\em Platonist} the issue of certain knowledge is literally far more important than for an Empiricist.  An empiricist in the end does not necessary believe as strongly that there are exact laws of Nature and absolute truth that we are attempting to discover; for an Empiricist, there are theories and lots of measurements of things.  For an Empiricist, the uncertainty is a matter of method and professionalism.  But for a Platonist, there is some Absolute truth, and so there is a hope of understanding how far are we from certain absolute knowledge of various secret feature of Nature herself.  Years ago, when I was working at Biospect/Predicant in San Francisco, I was just obsessed by the path to truth with the statistics.  

Now don't get me wrong.  I am fine with slapping down a $\sigma$ as the standard deviation just as much as the next guy saying 'within margin of errors' without any hint at all that I would be the sort of person who would get extremely excited if there is actually a sure theory to tell me some exact value with precision of 0.001 and so on.  In physics, I do not worry so much about exactness because our instruments are not capable of matching theory.  This seemingly pedantic philosophical issue is dominant in social sciences for me.  In social sciences the nature of noise is still not known precisely.  I am glad that De Moivre and others in 1750-1850 were confident that their sample size computations are exact and Gaussian noise is the actual noise in Nature.  I am glad because they were wrong, but they would have made no progress if they were stuck on this finicky issue.  Now I am superbly interested in perfection of measurement of Nature's true secret absolute truth values, because Four-Sphere Theory is correct.  But I am also glad I am working on Social Science problems rather than physics, because in physics, this is silly to do .  In Social Science the same sort of fussiness about exact noise is actually extremely substantial issue because there are methods of separating noise from signal and Social Science noise itself is still a mystery, and forget about Fermat, Pascal, Legendre, De Moivre, Laplace, but even Galton, Pearson, Edgeworth, and all the great statistical thinkers of twentieth century have not addressed this problem adequately and without resolving this, there is no real Social Science, because it's too easy to be cavalier and throw around a $\sigma$ here and there and pretend that there will be clarity in the mess of a wild anarchic orgy of statistical observations where we don't know what is confident certainty, what is exact noise, and whether the hand waving can ever stop.

\section{Zulf's Generalised Hyperbolic Noise Hypothesis}

For a long time, certainly from the late 1960s, it has been a popular issue is Science to seek non-Gaussian models of noise since Benoit Mandebrot was a student of Paul Levy and he loved stable self similar distributions.  On empirical grounds, I would advocate Ole Barndorff-Nielsen's Generalised Hyperbolic Disributuion.  Univariate fits of GHD give us location estime $\mu$ the location parameter.  It is crucial to consider $\mu(x)-\bar{x}$ as a serious issue because it can be large.  In this case what is the truth of nature?
I would propose in Social Science $\mu(x)$ is closer to absolute truth.

And this is the direction that inference might be improved in Social Science.

\end{document}