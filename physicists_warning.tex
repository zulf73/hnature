\documentclass{amsart}
\usepackage{graphicx}
\graphicspath{{./bn1976/}}
\usepackage{hyperref}
\usepackage{csvsimple}
\usepackage{longtable}
\usepackage{lscape}
\usepackage{epigraph}
\title{Zulf's Grave Warning To Physicists on Attempting To Reduce Psychology and Human Nature To Physics}
\author{Zulfikar Moinuddin Ahmed}
\date{\today}
\begin{document}
\maketitle
\section{The Key Issues}
Physics is absolutely beautiful science, and my Four-Sphere Model I worked on for more than a decade before it was thought right by Freeman Dyson.

However, I take a broad philosophical perspective about science and Nature.  The Human Race System has 7.8 billion brains, and each is more complex than will be possible to comprehend reduced to physics.  

\section{Parsimony Must Be Respected}

It is horrible decision to attempt to reduce Psychology to Physics because the brain is very complex.  And although the brain is governed by S4 Electromagnetic law, the complexity of the brain means that there needs to be a great many layers of sophistication before there is a reduction to Physics.  Well, Parsimony implies that we choose entirely different layers of concepts.  It would be an absolutely horrible science otherwise.  

In order for Psychology or Quantitative Human Nature to be beautiful science, Parsimony demands that we don't try to reduce it to physics directly.

\section{What Physicists Could Do}
If physicists want to work on these issues, abstract the techniques and ideas from Physics and do not attempt to reduce Psychology to Physics.  The latter would produce horribly {\em bad} science because it's impossible to gain insight without Parsimony, and Physics reduction is almost impossible without making messy and incomprehensible theories.  

Physics is a tool in Psychology, and should not drive the science. It would produce junk.

\section{My Hilbert Space Approach Principles}

Psychology's difficulty from a mathematical physics perspective (just macroscopic Four-Sphere no high energy) is that if you consider the body of a person as $K$ particles, not only is $K$ large, but you have gigantic number of potential partitions of $K$ that might produce subsystems, and then you introduce $N=10^{10}$ people in the world.  Suddenly the idea of reduction to physics dissipates because you can't produce a science this way.  Psychologists had taken a much more pragmatic approach and were conservative, declaring variables and subsystems by various other means.  Now their approach leads always to continuous confusions and controversies about foundations of psychology.  You just have to see twentieth century with B. F. Skinner's Behaviorism to Cognitive Revolution in the fifties etc.  The concepts of psychology continuously have to be re-examined and that is not a bad thing, because no one in their right mind with spend their precious lives working with partitions of $K$ particles of the body and expect any substance in the science.  So there is an art to picking variables.  There is a bootstrapping when more is discovered and concepts evolve.  None of it can be reduced to Physics concepts in any tractable manner.  

So what is good about my approach.  The particular class of Hilbert spaces I want to consider are restrictive as function spaces mathematically.  I don't mind putting even more mathematical constraints.  Regardless of how much mathematical constraints I put, the admissible psychological or social scientific variables is still always going to be too broad.  And that is the game I think one has to play, continuously tightening mathematical constraints and finding variables that actually have potential to be 'chosen of Nature'.  You see, the problem is that our psychological variables are our inventions and Nature may or may not bless them with great relations with other variables.  But my infrastructure is quite broad and general from social science perspective and still gives us more analytical tools and scientific models to produce coherence in what is understood.  Precision increases in Social Science without damaging the progress that has been achieved with great deal of talent and effort.  People in 'hard sciences' do not appreciate how precious each result is in social science because there are all manner of obstructions.  Social Science is actually very very hard.

\end{document}