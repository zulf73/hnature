\documentclass{amsart}
\usepackage{graphicx}
\graphicspath{{./}}
\usepackage{hyperref}
\usepackage{csvsimple}
\usepackage{longtable}
\usepackage{lscape}
\usepackage{epigraph}
\title{Aristotle's Virtue Ethics Theory is Successful} 
\author{Zulfikar Moinuddin Ahmed}
\date{\today}
\begin{document}
\maketitle

\section{Aristotle's Theory of Virtue}

Aristotle's theory of virtue as explicated in Nicomachean Ethics included the idea that virtues are developed through habituation, and that they lead to eudaimonia.

What I will show here is that eudaimonia is not Life Satisfaction.  I will do this by attention to certain moral values that involve {\em physical violence}.  I discovered that on some moral values involving violence, people who have strong convictions against use of violence have mildly lower Life Satisfaction than those who do not.

\section{Table}

% latex table generated in R 4.0.3 by xtable 1.8-4 package
% Sun Jun  6 13:13:43 2021
\begin{table}[ht]
\centering
\begin{tabular}{rlr}
  \hline
 & desc & satdiff \\ 
  \hline
1 & undeserved govt benefits & -3.72 \\ 
  2 & avoid paying public transport & -2.63 \\ 
  3 & stealing property & 1.02 \\ 
  4 & cheating on taxes & 2.74 \\ 
  5 & taking bribes on duty & 1.73 \\ 
  6 & homosexuality & -5.53 \\ 
  7 & prostitution & -0.95 \\ 
  8 & abortion & -1.15 \\ 
  9 & divorce & -3.13 \\ 
  10 & sex before marriage & -2.22 \\ 
  11 & suicide & 1.33 \\ 
  12 & euthanasia & -2.75 \\ 
  13 & man beating wife & 4.03 \\ 
  14 & beating children & 5.31 \\ 
  15 & violence against other people & 3.85 \\ 
  16 & terrorism as political, religious mean & 2.72 \\ 
  17 & casual sex & -1.40 \\ 
  18 & political violence & 3.02 \\ 
  19 & death penalty & -1.13 \\ 
   \hline
\end{tabular}
\end{table}

\section{Code}

\begin{verbatim}
% latex table generated in R 4.0.3 by xtable 1.8-4 package
% Sun Jun  6 13:23:01 2021
\begin{table}[ht]
\centering
\begin{tabular}{rlr}
  \hline
 & desc & satdiff \\ 
  \hline
1 & undeserved govt benefits & -3.72 \\ 
  2 & avoid paying public transport & -2.63 \\ 
  3 & stealing property & 1.02 \\ 
  4 & cheating on taxes & 2.74 \\ 
  5 & taking bribes on duty & 1.73 \\ 
  6 & homosexuality & -5.53 \\ 
  7 & prostitution & -0.95 \\ 
  8 & abortion & -1.15 \\ 
  9 & divorce & -3.13 \\ 
  10 & sex before marriage & -2.22 \\ 
  11 & suicide & 1.33 \\ 
  12 & euthanasia & -2.75 \\ 
  13 & man beating wife & 4.03 \\ 
  14 & beating children & 5.31 \\ 
  15 & violence against other people & 3.85 \\ 
  16 & terrorism as political, religious mean & 2.72 \\ 
  17 & casual sex & -1.40 \\ 
  18 & political violence & 3.02 \\ 
  19 & death penalty & -1.13 \\ 
   \hline
\end{tabular}
\end{table}
\end{verbatim}

\section{Aristotle is Wildly Successful}

I had various errors in interpretation of states for moral values previously.  Once these are corrected, we see something quite spectacular.  On morals involving violence, Aristotle is phenomenally successful in his theory.  I am quite overwhelmed at the moment, because I had the wrong conclusions for the past few days.  

\section{Zulf has No Apologies, Embarrassment, or Shame in Being Totally Wrong To Public}

I am a Great Scientific Genius.  I don't give a damn if I was wrong 90\% of the time and told the public totally erroneous things.  That does not make me ashamed or apologetic or embarrassed.  I find this quite comfortable to make errors and give my errors to the public.  Only inferior Scientists worry about this.


\section{Virtue Sets $A$ and $B$}

We can give a positive verdict for Aristotle's theory of virtues as follows.

First we identify 'eudaimonia' with life satisfaction.  Then we interpret his theory to be saying that if you possess virtues in some set $A$ then you will have higher eudaimonia. Our measurements this.

Let $A$ be the set of virtues regarding the following.  By 'virtue' we mean a person thinks doing this is wrong.

\begin{itemize}
\item{Stealing property}
\item{Cheating on Taxes}
\item{Suicide}
\item{Man beating wife}
\item{Beating children}
\item{Violence against others}
\item{Terrorism}
\item{Political violence}
\end{itemize}

Our empirical estimates show that for set $A$ of virtues Aristotle's theory is correct.  This is a spectacularly impressive theory because it is extremely nontrivial conjecture that moral virtues would actually lead to any positive advantage in life satisfaction.  This alone shows Aristotle's spectacular and unique genius.  

\section{Aristotle's Theory does not apply to Virtue list $B$}

Let thecomplement of $A$ be the virtue list $B$.  We can handle $B$ in several ways.  First, we could just declare the positions of $B$ to be not virtuous.  Then Aristotle's theory is verified with $A$ and we ignore $B$.  

We could also consider $-B$ to be the virtue and flip and then Aristotle will be universally correct with $-B$ as the right position. In the latter case we always have life satisfaction increasing with higher virtue.  This second position has merit.

\section{Hagiographic Material}

\includegraphics[scale=0.2]{aristotle.jpg}

I, Zulfikar Moinuddin Ahmed, am quite overwhelmed by the genius of Aristotle regarding {\em eudaimonia}.  Years ago, in 2008, I became quite interested in human well-being, and Plato's {\em Republic} and also Aristotle's Theory.  My work here represents one of the clearest empirical tests of Aristotle with global data.  Now Virtue ethics itself is much older than Aristotle, and Ancient Egypt had texts from 1300 BC, and also before Aristotle in East we had Confucius and Buddha as well promoting Virtue Ethics.  Regardless, {\em eudaimonia} was a precise concept that was promoted by Aristotle.  I am filled with wonder at this great genius who was actually right about improvement of Life Satisfaction as a universal feature of Human Beings, across the globe holding for 7.8 billion people of multiple ethnicities, religions and nations.  The profound nontrivial nature of his theory, when considered as a Scientific theory is just astounding.  I will have to admit that the confidence that was clear in the virtue theory that it will produce eudaimonia {\em has been difficult to believe} for 2400 years!  And he's right!  He's right about actual contemporary human beings in the billions.  This is genius that is magical.  Yes, he was wrong about some other things, but this alone fully justifies Aristotle as an immortal genius and a benefactor for human race of all ethnicities.


\end{document}