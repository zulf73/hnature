\documentclass{amsart} 
\usepackage{amsmath}
\usepackage{verse}
\DeclareMathOperator*{\argmax}{arg\,max}
\DeclareMathOperator*{\argmin}{arg\,min}
\usepackage{graphicx}
\graphicspath{{./}}
\usepackage[fontsize=14pt]{scrextend}
\usepackage{hyperref}
\usepackage{csvsimple}
\usepackage{epigraph}
\title{Why I am Quite Certain Rationality is Not Human Nature}
\author{Zulfikar Moinuddin Ahmed}
\date{\today}
\begin{document}
\maketitle

\section{I Don't Get Great Chess Rating}
I am not a serious competitive chess player at all.  I do amuse myself while dawdling and thinking about something else in the past decade on Chess Tactics Problems.  I am not able to get perfect scores at all.  I don't mind.  At 48, I am not horribly concerned about feeling that I have anything to prove regarding my intelligence, so I don't suddenly feel horribly inferior to much better chess players.  I am a Princeton graduate in Mathematics, and I am actually quite good at Reason.  I look at my errors a bit and realise that I actually need some practice to see the board more clearly and that requires {\em practice} which I am not willing to put effort into.  Without effort into practice, I make errors while my mind is floating around on things unrelated to chess.  That is clear evidence that I am not rational in the sense that there is no automaticity in my ability to just do perfectly logical steps that always beat the Chess Tactics problems.

\section{Simple Breakdown in Rationality Steps}

\subsection{Assumptions}
I am a trained mathematician, and so I am quite familiar with the following.  In order to come to any rational conclusions, we need assumptions and axioms, and then we can come to conclusions by deduction by mathematical deduction.  Now human beings have our primary beliefs that substitute for axioms.  We consider these self-evident in some cases, or we pick them up from Holy Books as Divine Truths, from Bible or Quran.  Or we pick them up from experience.  Or we believe some source of information.  In one way or other, for rationality we need to understand our source of axioms and fundamental assumptions.  We do not all share the same beliefs at all.  One of my great achievements has been to find that across the human race we do have remarkable {\em agreement} on many things regardless, such as justifiability for stealing being exponentially distributed with vast majority considering it wrong.  But there is a stochastic model for these as well.  If human beings do not share axioms and assumptions, then obviously even perfect ability for logical deduction would lead to different 'rational conclusions'.

What drives the acceptance of axioms is faith and trust.  These have a psychological dynamic of their own.

\section{Lack of Uniformity in Correct Deductions}

Given exactly the same assumptions human beings vary in our ability to get the correct logical conclusions.  Some of us are very good at unerringly reaching the correct conclusion, and some of us find logical deduction from assumptions to be well-nigh impossible.  This then puts how much "Reason is Human Nature" can be true.  

\section{We are intuitive beings in most cases}

Daniel Kahneman's Thinking Fast and Slow tells us that there is a fast intuitive Level 1 thinking that is fast but error-prone.  This is Human Nature.  We have some processes to do approximately correct natural conclusions.  

\section{Socrates was speculating}

It was Socrates who introduced the notion that Human Beings are Rational by Human Nature.  This was speculation, as consequential as this might seem.  This notion is the successful propaganda of Socrates, or successful religious faith of Socrates.  Even Charles Darwin was quite faithful to the notion that Reason is one of our highest faculty.  

\section{Reason is fabulous but not Human Nature}

Reason is truly magnificent in Science.  But it is most certainly not Human Nature.  I am certainly not suggesting that we are 'lower' than hypothetical beings for whom Reason is {\em their} nature.  Reason is an option and a fruitful one for some of us to gain great talent; but it is not actually part of our Nature, and it was never part of our Human Nature.

\section{Not Reason but What?}

This is the great mystery, which is to understand what sort of cognitive processes are actually Human Nature if not Reason.  Reason is quite attractive; it was attractive as a principle for Immanuel Kant.  

\section{Problem of assuming Reason is Human Nature}

Human Race is a Great and Noble Race regardless of whether Reason is our Human Nature.  But if Reason is not part of our nature, then we will constantly be making errors that will produce ill-being across the world all the time if we act as though Reason were Human Nature because it's just not true.

 
\end{document}