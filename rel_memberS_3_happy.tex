\documentclass{amsart} 
\usepackage{amsmath}
\usepackage{verse}
\DeclareMathOperator*{\argmax}{arg\,max}
\DeclareMathOperator*{\argmin}{arg\,min}
\usepackage{graphicx}
\graphicspath{{./}}
\usepackage[fontsize=14pt]{scrextend}
\usepackage{hyperref}
\usepackage{csvsimple}
\usepackage{epigraph}
\title{Religious Organisation Memebership has Weak Positive Effect on Happiness}
\author{Zulfikar Moinuddin Ahmed}
\date{\today}
\begin{document}
\maketitle

Percentage of world membership in religious organisations have a positive effect on Happiness but explains only 2.5 percent of variation.

\section{Result}

\begin{verbatim}
> summary(mod.relhap)

Call:
lm(formula = log(hap + 1) ~ log(actRel + 1), data = relhap)

Residuals:
     Min       1Q   Median       3Q      Max 
-0.47539 -0.04287  0.03187  0.06401  0.14922 

Coefficients:
                Estimate Std. Error t value
(Intercept)      4.40191    0.03986 110.443
log(actRel + 1)  0.01609    0.01455   1.106
                Pr(>|t|)    
(Intercept)       <2e-16 ***
log(actRel + 1)    0.274    
---
Signif. codes:  
0 ‘***’ 0.001 ‘**’ 0.01 ‘*’ 0.05 ‘.’ 0.1 ‘ ’ 1

Residual standard error: 0.1092 on 49 degrees of freedom
Multiple R-squared:  0.02436,	Adjusted R-squared:  0.004451 
F-statistic: 1.224 on 1 and 49 DF,  p-value: 0.2741
\end{verbatim}


\section{Interpretation}

Membership in a religious organisation is less effective in improving happiness than trust in most people, financial satsfaction and autonomy.  However,  it is {\em positive}.  I will comment that I have seen numerous articles about whether religious people are happier or not.  We claim our results are stronger than these because $N=51$ and the quality of our data are good.  Our results are Human Race oriented rather than national.

\end{document}
