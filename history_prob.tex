\documentclass{amsart}
\usepackage{graphicx}
\graphicspath{{./}}
\usepackage{hyperref}
\usepackage{csvsimple}
\usepackage{longtable}
\usepackage{lscape}
\usepackage{epigraph}
\title{Probability Ideas from 1660s to 2021} 
\author{Zulfikar Moinuddin Ahmed}
\date{\today}
\begin{document}
\maketitle

\epigraph{
He deals the cards as a meditation
And those he plays never suspect
He doesn't play for the money he wins
He doesn't play for respect

He deals the cards to find the answer
The sacred geometry of chance
The hidden laws of a probable outcome
The numbers lead a dance

I know that the spades are the swords of a soldier
I know that the clubs are weapons of war
I know that diamonds mean money for this art
But that's not the shape of my heart
}{Sting, "The Shape of My Heart"}

I probably have said this before, but I disliked probability theory quite a bit when I was young.  All those combinatorial calculations involving shuffling decks and other games of chance I could not stand.  Even after college at Lehman Brothers Fixed Income Division, I was not particularly fond of probability theory.  I remember getting more excited about quasirandom numbers.  I liked then the idea that with $1/\sqrt{N}$ accuracy you could estimate integrals by random sampling in Monte Carlo.  The latter seemed more respectable.  But I had deeper problems with stochasticity.  I had such great problems that even though I loved the various fractal images, and Benoit Mandelbrot used to give some talks at Ohio State University Ross Program, that even years later I was not all that convinced that Riemann's ideas were wrong and revived them with Four-Sphere Theory because I am classical and I am still promoting a purely deterministic physics. I am right, of course, but I am not right by {\em chance} but because I am right.  Stochasticity at the heart of all things of Nature is an abomination.  I know Albert Einstein's comments about God does not play dice with the universe; but I also don't think God bends time in order to keep speed of light constant.  I am pretty conservative and was appalled by expansion of space as physical law.  It is totally absurd and is not true.  In truth the universe is a static eternal four-sphere with rigid metric radius $R=3075.69$ Mpc whose surface is the medium of propagation of light.  

But I digress.  I am attempting to understand some of the history of probability theory.  The idea arose strongly around 1660, various issues of games, probabilities, decisions.  I am extremely classical about science, and so I am not all that enthusiastic about things that I consider frivolous.  I don't want the issues of Scientific Inference to be confused by ideas from games of chance and other more frivolous parlour games.  Things are quite a bit more serious when Scientific Inference is concerned.  Here the issue is that we, Man, would like to produce precise theories about how existence, the world in which we live, operates and we consider these operating principles to be the way in which Nature actually operates.  In other words, we want to be enormously ambitious and although we are continually shown how bad we are at deciphering even the most simple primitive secrets of Nature, even though we are not at all at ease in the interpreted world and seek escape through lovemaking and death--if you are Rilke and have the sensitivity for such things--or by deciding that although life and Nature are too beautiful to comprehend we will build nuclear bombs and destroy all sorts of civilian cities in Japan and pretend to be great benefactors of humanity, and to top it off claim that "I am become death; I am become Shiva the Destroyer of Worlds" if you are more Robert Oppenheimer.





\end{document}
