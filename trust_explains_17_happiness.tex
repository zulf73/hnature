\documentclass{amsart} 
\usepackage{graphicx}
\graphicspath{{./}}
\usepackage[fontsize=14pt]{scrextend}
\usepackage{hyperref}
\usepackage{csvsimple}
\usepackage{epigraph}
\title{Trusting Nature explains 16.5\% of Happiness Variation in Humans}
\author{Zulfikar Moinuddin Ahmed}
\date{\today}
\begin{document}
\maketitle
\epigraph{All I need is one

One old man is enough

Babe, you got it wrong

Just turn your fears into trust, to trust}{LSD, "Thunderclouds"}


\section{Results}

\begin{verbatim}
> summary(mod.haptrust)

Call:
lm(formula = Happy ~ trustMost, data = haptrust)

Residuals:
     Min       1Q   Median       3Q      Max 
-28.9996  -3.5870   0.7568   4.4686  13.0354 

Coefficients:
            Estimate Std. Error t value Pr(>|t|)    
(Intercept) 80.51397    1.44953  55.545   <2e-16 ***
trustMost    0.18364    0.04701   3.906    2e-04 ***
---
Signif. codes:  0 ‘***’ 0.001 ‘**’ 0.01 ‘*’ 0.05 ‘.’ 0.1 ‘ ’ 1

Residual standard error: 7.473 on 77 degrees of freedom
Multiple R-squared:  0.1654,	Adjusted R-squared:  0.1546 
F-statistic: 15.26 on 1 and 77 DF,  p-value: 0.0001998
\end{verbatim}

\section{Interpretation}
With $N=79$ countries data from World Values Survey, we can be extremely confident $p=0.0002$ that trusting all people will explain 16.5\% of the variance of happiness.  This implies that trusting people will make you happier.  This is not a guess.  It is {\em human nature}.  Trusting all people is not enough, but do turn your fears into trust more often.

\end{document}
