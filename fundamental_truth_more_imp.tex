\documentclass{amsart}
\usepackage{graphicx}
\graphicspath{{./}}
\usepackage{hyperref}
\usepackage{csvsimple}
\usepackage{longtable}
\usepackage{lscape}
\usepackage{epigraph}
\title{Issues of Fundamental Truth and Purpose of Existence More Rather Than Less Important for Human Race as Religions Falter} 
\author{Zulfikar Moinuddin Ahmed}
\date{\today}
\begin{document}
\maketitle

\section{Personal Spiritual Journey}

I will discuss later the issues of vindication of Aristotle on increase of Life Satisfaction for certain virtues as well as the decrease of life satisfaction for religious beliefs.

These are issues that have been central to me my entire life. At least for the major portion of my life 1979-2008 I have been empiricist scientifically minded Atheist.  I want to restrict attention to concerns from this period of my life which is closer to most people's experience because I have developed unique religious understanding afterward that do not necessarily have relevance for most people on Earth.

From my point of view the crisis of religion really began during the First World War, as the massive bloodshed around the world brought into question the relevance of God.

I can prove that the actual universe does not follow Big Bang and Expansionary Cosmology but instead is Eternally Recurrent by Four-Sphere Theory.

Now I will tell you some ideas of the {\em purpose} of Man's invention of God.  First of all, let me be specific.  The Monotheistic God was an invention of Man.  This is the God that has anthropomorphic features.  The invention was necessary for survival of Civilisation, for a repository of Authority.  The various God-Kings of Roman Empire in the West and everywhere else also were intolerable and not very divine in their corruption and madness and separating Authority to a higher power with various qualities was necessary.  I was an Atheist with regard to {\em this} concept of God since 1979.  My aunt, mother, grandmother etc. are all Muslim, so I went my own way.  

However, there is something else that is quite different.  I do have reverence for Nature and the inscrutible and mysterious way that Nature works.  I was never under any delusion that Human Beings had any special right or ability to discern the laws of Nature; I knew this was a task that would always be imperfect.

Religions have always arose from the wisdom and myths of a culture or civilisations that developed over historical time and represented the repository of understanding the world and finding coherence in our rituals to conform to it.  I won't go much into the process.  Now my serious viewpoint about this is that these are Man's discoveries over time and they all have depth and merit that which became part of the religion.  

This need to have mythological understanding of {\em existence} and our lives within it is to me quite clearly fundamental human nature.  I am not particularly impressed with Jean-Paul Sartre and the Existentialists and the Absurdists and so on because they are not serious; they are not able to provide my Beloved People, the Human Race with a coherent and rich world that is accurate and in accordance with our best wisdom regarding ourselves that can provide sane and meaningful lives.  

We are now in various "post" world, post-colonial, post-industrial, post-modern, post-this, post-that. These organisations do not satisfy at all.  For me, at least, religions did not compel for very good reasons; they did not give me any credible answers that were compelling to my soul.  Now I do believe that Human Race is an Angelic Race, and that there is beautiful and natural order that is not clear in the confusion of people's theories that are not sharp and true.  I see therefore a moment for a new Dawn of Civilisation right now, an opportunity to gain clear understanding of truth.  

Religions are not going anywhere fast.  Religions like Christianity and Islam just absorb everything in their path and so are essentially here to stay.  But what is possible is to have renewed feeling for the actual mystery of Nature, for Nature is vastly unknown and get past the mostly uninteresting issue of whether the phenomena is "God's Will" or they are not.  Who cares?  I would rather what exactly was quickly attributed to God's Will more carefully.  For me Fate is due to S4 Electromagnetism, and a vast ocean beyond the Galaxies filled with electromagnetic life just as complex as Pacific Ocean.  If you want to call the currents in this Ocean "God's Will" you are quite bold, because this sort of God is not going to be bringing you any presents through the chimney exactly.  I call it Fate.

\section{The Numbers Again}

I'll give you numbers based on Life Satisfaction.

% latex table generated in R 4.0.3 by xtable 1.8-4 package
% Mon Jun  7 15:39:28 2021
\begin{table}[ht]
\centering
\begin{tabular}{rlr}
  \hline
 & bdesc & sd \\ 
  \hline
1 & imp god & -3.24 \\ 
  2 & blf god & -1.04 \\ 
  3 & blf afterlife & -1.12 \\ 
  4 & hell & -1.77 \\ 
  5 & heaven & -2.28 \\ 
  6 & rel$>$sci & -4.42 \\ 
  7 & my relig right & -1.04 \\ 
  8 & freq relig service & -1.73 \\ 
  9 & freq pray & -2.95 \\ 
  10 & relig person & -2.16 \\ 
  11 & follow relig rules or do good & 0.05 \\ 
  12 & here vs after death & -5.06 \\ 
  13 & confusion moral rules & -5.89 \\ 
   \hline
\end{tabular}
\end{table}


Now let me show you the Aristotelian part on Virtues.  Note that Virtues are human issues; the Ancient Greeks did produce virtues to divinity, various demi-gods etc.   Regardless, actual people have Virtues independent of mythology.  This is what Martin Seligman, Christopher Peterson and others had shown in their work.  What I showed recently is that Aristotle is vindicated by a number of virtues.

% latex table generated in R 4.0.3 by xtable 1.8-4 package
% Sun Jun  6 13:13:43 2021
\begin{table}[ht]
\centering
\begin{tabular}{rlr}
  \hline
 & desc & satdiff \\ 
  \hline
1 & undeserved govt benefits & -3.72 \\ 
  2 & avoid paying public transport & -2.63 \\ 
  3 & stealing property & 1.02 \\ 
  4 & cheating on taxes & 2.74 \\ 
  5 & taking bribes on duty & 1.73 \\ 
  6 & homosexuality & -5.53 \\ 
  7 & prostitution & -0.95 \\ 
  8 & abortion & -1.15 \\ 
  9 & divorce & -3.13 \\ 
  10 & sex before marriage & -2.22 \\ 
  11 & suicide & 1.33 \\ 
  12 & euthanasia & -2.75 \\ 
  13 & man beating wife & 4.03 \\ 
  14 & beating children & 5.31 \\ 
  15 & violence against other people & 3.85 \\ 
  16 & terrorism as political, religious mean & 2.72 \\ 
  17 & casual sex & -1.40 \\ 
  18 & political violence & 3.02 \\ 
  19 & death penalty & -1.13 \\ 
   \hline
\end{tabular}
\end{table}



\end{document}