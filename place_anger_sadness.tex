\documentclass{amsart} 
\usepackage{amsmath}
\DeclareMathOperator*{\argmax}{arg\,max}
\DeclareMathOperator*{\argmin}{arg\,min}
\usepackage{graphicx}
\graphicspath{{./}}
\usepackage[fontsize=14pt]{scrextend}
\usepackage{hyperref}
\usepackage{csvsimple}
\usepackage{epigraph}
\title{Thoughts About the Place of Anger and Sadness}
\author{Zulfikar Moinuddin Ahmed}
\date{\today}

\begin{document}
\maketitle

\section{Introduction}

The prevailing view about Anger and Sadness are primarily negative.  Panksepp's Affective Neuroscience sees them as fundamental systems, and evolutionary biologists consider them necessary to react to threats.  

I want to outline a slightly different way of seeing the world.  Anger and Sadness, in this direction are not ancient evolutionary adaptations that are no longer necessary except to keep under wraps.

What is much more interesting is to consider both to be tamed by development of Virtuous Character so that they might be tuned to play a strictly positive role in our life experience.  Anger and Sadness are such fundamental systems of the Brain that we ought to consider them indispensable by definition of human being.  Our evolution ought to have honed the place of these emotions in the actual world that exists today.   


\end{document}