\documentclass{amsart}
\usepackage{graphicx}
\graphicspath{{./}}
\usepackage{hyperref}
\usepackage{csvsimple}
\usepackage{longtable}
\usepackage{epigraph}
\title{Extreme Terrorism Support Totally Independent of Happiness and Trust Level}
\author{Zulfikar Moinuddin Ahmed}
\date{\today}
\begin{document}
\maketitle

We report here that across the world, those whos have extreme support of terrorism have almost exactly the same happiness level and trust profile as the general population.

This is a new result and puts constraints on our understanding for support for terrorism.  We divide up the world into six ethnic classes but we don't worry about the ethnicity for this result.

\section{Happiness Profile of the General Population}

% latex table generated in R 4.0.3 by xtable 1.8-4 package
% Mon May  3 18:25:15 2021
\begin{table}[ht]
\centering
\begin{tabular}{rrr}
  \hline
 & 1 & 2 \\ 
  \hline
1 & 6.68 & 25.96 \\ 
  2 & 12.25 & 40.62 \\ 
  3 & 1.66 & 10.46 \\ 
  4 & 0.22 & 2.15 \\ 
   \hline
\end{tabular}
\end{table}

\section{Happiness and Trust of Supporters of Terrorism}

% latex table generated in R 4.0.3 by xtable 1.8-4 package
% Mon May  3 18:26:17 2021
\begin{table}[ht]
\centering
\begin{tabular}{rrr}
  \hline
 & 1 & 2 \\ 
  \hline
1 & 7.28 & 32.73 \\ 
  2 & 9.95 & 33.20 \\ 
  3 & 2.11 & 11.51 \\ 
  4 & 0.23 & 2.98 \\ 
   \hline
\end{tabular}
\end{table}

\section{Conclusion}

This is a basic independence result.  Some ideas about what produces support for political violence is justified at 9 or 10 might posit some issues with trust level or happiness level of the individuals.  We can safely conclude that both Happiness and Trust levels are not significant correlates for high support for political violence.


\end{document}
