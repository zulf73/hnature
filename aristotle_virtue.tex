\documentclass{amsart}
\usepackage{graphicx}
\graphicspath{{./}}
\usepackage{hyperref}
\usepackage{csvsimple}
\usepackage{longtable}
\usepackage{lscape}
\usepackage{epigraph}
\title{Empirical Test Of Aristotle's Theory of Virtue For Stealing} 
\author{Zulfikar Moinuddin Ahmed}
\date{\today}
\begin{document}
\maketitle

\section{Results}

Let $S$ stand for satisfied with life; let $H$ stand for happy; let $V$ stand for virtuous and $NV$ nonvirtuous.  

We have the following empirical estimates from World Values Survey data.
\[
P(S|V) = 0.21
\]
while
\[
P(S| NV) = 0.227
\]
We also have
\[
P( H | V ) = 0.8655
\]
and
\[
P( H | NV ) = 0.873
\]
These contradict Aristotle's theory that virtues lead to happiness.

\section{Violence Against Others}

\begin{eqnarray*}
P(H|V) &= 0.8667\\
P(H|NV) &= 0.8607\\
P(S|V) &= 0.214\\
P(S|NV) &= 0.253
\end{eqnarray*}

In this case we see that Aristotle is right for $H$ but not for $S$.

\section{Justifiable for Man to Beat Wife}

\begin{eqnarray*}
P(S|V) &= 0.215\\
P(S|NV) &= 0.244\\
P(H|V) &= 0.867\\
P(H|NV) &= 0.851
\end{eqnarray*}

\section{Return To Aristotle's Theory}

Aristotle's theory is that one ought to develop virtues in order to gain happiness in the sense of {\em eudaimonia}.  Our last two examples had some flaws.  We are using {\em moral convictions} as proxy for virtues, which is not strictly Aristotle's theory.  In fact, I was the pioneer recently of a new interpretation of {\em phronesis} which I interpret as an internal development rather than external one.

Regardless, now we have a sense for the sorts of numbers involved in massive global measurements of happiness and we can see that Aristotle's theory will be quite subtle and difficult to understand in precision of scientific truth.  Moral conviction effects on happiness and life satisfaction do not seem extremely strong to us.

\section{What Our Data Suggests}

Our data suggests that there is a mild increase in happiness and a mild decrease in life satifaction when we condition on moral convictions being good or not.  Moral convictions could be quite different from actual virtues so this regularity is not exactly able to test Aristotle's actual claim regarding eudaimonia but these are real results from data and so we can consider that happiness might not be a function of moral convictions at all.  

\section{Tests of Actual Virtuous Character versus Happiness and Life Satisfaction Quite Open}

We want to emphasise that there is very little clear sense of what actually produces happiness or life satisfaction. Moral values in conviction may not actually reflect effect of a virtuous character.

\end{document}