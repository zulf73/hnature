\documentclass{amsart}
\usepackage{graphicx}
\graphicspath{{./}}
\usepackage{hyperref}
\usepackage{csvsimple}
\usepackage{longtable}
\usepackage{lscape}
\usepackage{epigraph}
\title{On The Bill Gates Question} 
\author{Zulfikar Moinuddin Ahmed}
\date{\today}
\begin{document}
\maketitle

\section{Prologue}

Our story begins sometime early in 2020.  At this time, Bill Gates had everything.  He had tremendous success throughout his life, from early 1970s.  He had wealth, around \$130.5 billion. He was married to a beautiful woman.  He had three children.  He had many mansions, and yachts and cars.  He had respect of society.  He was the envy of all people around the world.

Then we have a cutscene and we see Bill Gates being gassed in Auschwitz, with safety glasses to keep the maddening crowd from being affected and see him with alarmed eyes, the gas seeping into the pores of his body, the cameras and the crowd gawking with HDTV showing his horrific and pitiful death scene to billions around the world.

Somewhere in Bengal, a man put on Ajoy Chaterjee's version of Kazi Nazrul Islam's Song, "Chirodin Kaharo Shomano Nahi Jai".

\section{Zulf}

Zulf was enormously filled with a sense of his own sacredness and thought of himself as the Soul of an Archangel in the end.  He was conscientious about treating people of all walks of life with dignity and respect, when he could afford it, but he was also exceedingly finicky about sexual engagement with women who were not just right and was quite private.  He was murderously enraged at Bill Gates for desecrating private spaces by the horrific disgusting abomination and white supremacist disgusting matter.  It was only inevitable that his vengeance would be quite public for insolence that was intolerable, from the vastly inferiour Bill Gates.


\end{document}