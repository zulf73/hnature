\documentclass{amsart} 
\usepackage{graphicx}
\graphicspath{{./}}
\usepackage[fontsize=14pt]{scrextend}
\usepackage{hyperref}
\usepackage{csvsimple}
\usepackage{epigraph}
\title{Towards a Mathematical Theory of Human Nature}
\author{Zulfikar Moinuddin Ahmed}
\date{\today}
\begin{document}
\maketitle


\section{Human Nature as Constant Functions}

We represent human race as a set $\mathcal{H}$ with 7.8 billion elements.  These represent all people alive on Earth.
We consider functionals $F:\mathcal{H}\rightarrow \mathbf{R}$ as measurements for each human being.

We don't care too much about {\em meaning} of the functionals $F$.  They could be anything with valid measurement possibility. We won't worry about topology here.  We can consider the metric topology on $\mathbf{R}$ and not worry too much about the topology of $\mathcal{H}$.  

\section{Human Nature Candidate Functions}

My nomenclature is provisional for "Human Nature Candidates" denoted HMC.  We consider all constant functions certainly to be candidates of Human Nature.  In addition to constant functions we want to consider those with Generalized Hyperbolic distributions to be HMC.

The reason for this definition is that there is a severe restriction in these functions mathematically and there is tractability of a small set of distributions that is non-trivial from the mathematical point of view.  In other words, even though from humanities and social science view, this framework is too broad and abstract, the restriction is highly nontrivial mathematically, and therefore this leads to a non-trivial scientific framework for the study of precise science of Human Nature.

\section{Overcompensation for Past Restrictions?}

From Ancient Mesopotamia to Egypt to China and India to Greece there have been general theories of Human Nature.  In 17th-18th century Europe there have been as well.  They all suffered in my opinion on the problem of overspecification, of overfitting to particular small populations.  I have decided the major problem is not to have a valid analytical infrastructure that is mathematically precise.  Our solution here is not to rush into specific questions of interest but to build infrastructure for putting together a theory of Human Nature that can be coherent and tractable and parsimonious.

\section{Quantitative Human Nature is Possible}

With scientific evidence of single human race descended from a frozen tribe geographically concentrated in East Africa 75,000 years ago implies that a mathematical theory of Human Nature could provide us fundamental value for all our political struggles and advance human race to a new moment in our history, and could provide basis for simple understanding of a rich Human Nature.

Our framework makes it easy to determine whether something is Human Nature or not.  Any functional $F:\mathcal{H}\rightarrow\mathbf{R}$ can be checked by its measurement on actual people and mathematical criteria can determine if $F$ is human nature or not.  This evades all assumptions about Human Nature that are wrong.  If any measurement shows simple patterns across the human race in terms of smoothness and small number of parameters then it is interesting as candidate for human nature. 

This is so because we have 'integrated' all contentious issues of race, ethnicity, language, culture and are left with clean view of the patterns of the entire human race.  The point of view can allow us to appreciate what is truly unique for us, what is culture-specific and so on.

\end{document}