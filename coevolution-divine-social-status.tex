\documentclass{amsart} 
\usepackage{graphicx}
\graphicspath{{./}}
\usepackage[fontsize=14pt]{scrextend}
\usepackage{hyperref}
\usepackage{csvsimple}
\usepackage{epigraph}
\title{Co-Evolution of Divine and Social Status in Early Ancestors}
\author{Zulfikar Moinuddin Ahmed}
\date{\today}
\begin{document}
\maketitle

\section{Modern People have all sorts of odd Beliefs}

The assumption that modern people are somehow qualitatively different from our ancestors 300,000 years ago is wrong.  In terms of our beliefs both in West and East we find all manner of odd things that are held widely on faith.  Around half Americans recently did not want Arabic numerals taught in school because of their negative view of Arabs.  

What is called 'religion' today was a concept in the Western literati from 17th and 18th centuries.  

I will propose that 300,000-75,000 years ago all our ancestors in East Africa were no different.  They believed in all sorts of things, and that is what religion was then.  The division of what would be sacred and what would be profane was a much later development.  So when we say 'origin of religion' we can safely consider the coherent worldview including all sorts of beliefs that were shared by our ancestors in the Lower and Upper Paleolithic Era as 'religion'.  

Now Neanderthals are known to have buried their dead, so we can for the sake of parsimony assume that they had a collective worldview including the living and the dead, the natural and the supernatural and this was religion.  In particular I want to propose that social status of men or women in human society co-evolved with notions of the divine.  

In particular I want to hypothesize that more divine people occupied higher social status, and the collective religions were tied to the social status.

\section{Confusion of theories of Origins of Religions}
I will just put here the Wikipedia entry because it is valuable to gain context.

The earliest archeological evidence of religious ideas dates back several hundred thousand years to the Middle and Lower Paleolithic periods. Archaeologists take apparent intentional burials of early Homo sapiens and Neanderthals from as early as 300,000 years ago as evidence of religious ideas. Other evidence of religious ideas includes symbolic artifacts from Middle Stone Age sites in Africa. However, the interpretation of early paleolithic artifacts, with regard to how they relate to religious ideas, remains controversial. Archeological evidence from more recent periods is less controversial. Scientists generally interpret a number of artifacts from the Upper Paleolithic (50,000-13,000 BCE) as representing religious ideas. Examples of Upper Paleolithic remains associated with religious beliefs include the lion man, the Venus figurines, cave paintings from Chauvet Cave and the elaborate ritual burial from Sungir.

In the 19th century researchers proposed various theories regarding the origin of religion, challenging earlier claims of a Christianity-like urreligion. Early theorists Edward Burnett Tylor (1832-1917) and Herbert Spencer (1820-1903) emphasised the concept of animism, while archaeologist John Lubbock (1834-1913) used the term "fetishism". Meanwhile, religious scholar Max Müller (1823-1900) theorized that religion began in hedonism and folklorist Wilhelm Mannhardt (1831-1880) suggested that religion began in "naturalism" – by which he meant mythological explanation of natural events. All of these theories have since been widely criticized; there is no broad consensus regarding the origin of religion.

\section{Advantages of My Hypothesis}

Let us denote by "Early Ancestors" as our ancestors in East Africa in the period 300kya-75kya. It is extremely highly unlikely that without writing and other innovations that Early Ancestors had any need for separation of sacred and profane in their beliefs and coherence in collective worldview.  Therefore it is unlikely that the separation of the divine from the secular thought could be distinguished uniformly in East Africa 300kya-75kya.  Therefore it is not beneficial to separate origin of 'religion' from the rest of the comprehensive life and thought of Early Ancestors.  Therefore I consider the co-evolution of divine and social status as convergent features of the evolution of the collective psyche.

\section{Co-Evolution of Divine and Social Status is Genuinely New}

The hypothesis of Co-Evolution of Divine and Social Status is new and I, Zulfikar Moinuddin Ahmed, am the pioneer for this hypothesis.  I want full credit for pioneering this hypothesis and it does not have any previous occurrence in the literature.  I am confident that this is correct, so I do not want anyone else to claim priority over me for it.  It is my great insight.


\end{document}
