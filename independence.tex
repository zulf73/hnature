\documentclass{amsart}
\usepackage{graphicx}
\graphicspath{{./}}
\usepackage{hyperref}
\usepackage{csvsimple}
\usepackage{longtable}
\usepackage{epigraph}
\title{Independence Results For Religious Attitudes}
\author{Zulfikar Moinuddin Ahmed}
\date{\today}
\begin{document}
\maketitle

\section{Independence of Variables}

In probability theory, independence is $P(AB)=P(A)P(B)$. We will just consider this directly on some variables from World Values Survey.

\section{Correction to Independence Test for Categorical Variables}

\begin{verbatim}
g<-function(x){
  log(-log(x))
}

zulf.sigma<-function(z){
  n<-length(z)
  D<-matrix(0, nrow=n, ncol=n)
  s<-0
  for (j in 2:n){
    for (k in 1:(n-1)){
      gg = abs( z[j] - z[k])
      D[j,k] <- gg
      D[k,j] <- gg
      s<-s + gg
    }
  }
  out<- 2*s/(n^2-n)
  out
}

zulf.chisq<-function( data ){
  m<-dim(data)[1]
  n<-dim(data)[2]
  t<-0
  v0<-g(data[1,1:(n-1)]/sum(data[1,1:(n-1)]))
  sigma0 <- zulf.sigma( v0 )
  mu0 <- mean(v0)
  for (j in 2:m){
    v<-g(data[j,1:(n-1)]/sum(data[j,1:(n-1)]))
    sigma <- zulf.sigma( v )
    mu <- mean(v)
    t<- t + sum( ((v-mu)/sigma-(v0-mu0)/sigma0)^2)
  }
  df<-(n-1)*(m-1)
  t0<-qchisq(0.95, df=df)
  pval <- 1 - pchisq( t, df=df)
  list(tstat=t,pval=pval,crit=t0)
}
\end{verbatim}

\section{Importance of religion in Life and My Religion is only right religion are independent}

I am examining the appropriate statistical test for independence in this situation.  I invented a chisquare test that is different from the standard Pearson chisquare test of independence.

\begin{verbatim}
> zulf.chisq(rtbl)
$tstat
[1] 2.158791

$pval
[1] 0.9887003

$crit
[1] 16.91898
\end{verbatim}

The null hypothesis is not rejected here and the variables are independent.



This tells us that the rest of the rows have proportions too close to the first row to reject the null of all rows have same proportions.


\end{document}