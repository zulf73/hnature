\documentclass{amsart} 
\usepackage{amsmath}
\usepackage{verse}
\DeclareMathOperator*{\argmax}{arg\,max}
\DeclareMathOperator*{\argmin}{arg\,min}
\usepackage{graphicx}
\graphicspath{{./}}
\usepackage[fontsize=14pt]{scrextend}
\usepackage{hyperref}
\usepackage{csvsimple}
\usepackage{epigraph}
\title{Zulf's Theory of Respect}
\author{Zulfikar Moinuddin Ahmed}
\date{\today}
\begin{document}
\maketitle

\section{I want to be Respected Everywhere}

Wherever I go in the entire world, I want to be respected.  That's a psychological need of mine.  My philosophy of Respect is quite elementary.  I want to be respected and I can't tolerate people disrespecting me.  I can't stand Bill Gates and want him dead because he is disrespectful and worthless and does not have achievements equal to my pinky finger.  So I don't tolerate these insolent sons of bitches like Bill Gates.

Now I can generalise my view, and consider that I am willing to respect everyone in the world until they give me reason to disrespect me, like the case of Bill Gates.  

\section{Respect for Authority}

Unless I am considered Authority and am respected for that reason, I don't care.  I want to be respected, and I am willing to respect anyone at all if they are respectful.  People ought to be given a small window of opportunity to find if they are respectful or not, and if not they ought to be shunned.

\end{document}
