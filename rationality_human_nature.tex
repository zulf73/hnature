\documentclass{amsart} 
\usepackage{graphicx}
\graphicspath{{./}}
\usepackage[fontsize=14pt]{scrextend}
\usepackage{hyperref}
\usepackage{csvsimple}
\usepackage{epigraph}
\title{Reason and Rationality are Great but are they Human Nature?}
\author{Zulfikar Moinuddin Ahmed}
\date{\today}
\begin{document}
\maketitle

\epigraph{Have I been wrong

Have I been wise to shut my eyes and play along

Hypnotized paralyzed by what my eyes have found

By what my eyes have seen
What they have seen

Have I been blind

Have I been lost have I been wrong

Have I been wise have I been strong

Have I been hypnotized mesmerized
By what my eyes have found}{Natalie Merchant, "Carnival"}

To my philosopher friends, I would posit that you are doing something fabulous and highly derived, but if you tried to live as a philosopher 3 million years ago in East African savannah with Pure Reason you would be food for a sabre tooth cat.  Reason, my friends, is a tremendous advance for the Human Race, but it is not natural part of our Human Nature.


Before 75kya for 6 million years human ancestors were adapting to all sorts of problems and surviving and prospering with bigger brains of the babies.  For six million years, much longer than the 75kya that we've been out of Africa.  I am just trying to make sense of what sort of rationality grows in this situation.

\section{An anecdote from 2008}

You know when I was homeless in New York City, once I found a nice cardboard box and fell asleep in a nook in some neighborhood.  Well in the middle of the night, I got woken up with a bunch of white people claiming that this was their turf and I could not sleep there.  One of them had shaved head and they were pretty much convinced that the two men and women were Nazi Master Race themselves with global supremacy for cardboard sidewalk sleeping.  I looked at them and sized up without any experience that they would kill me without remorse and somehow I just managed to fanagle my way out of not getting killed.  I am  Princetonian used to lofts and \$150-200k salaries so how did I do that?  

I can tell you the answer.  Some of the adaptation of 6 million years of survival in East Africa kicked in the moment I realized that I had danger of death.  I won't go into the diaries of Adventures of Hobo Zulf.  The point is that we carry in our genes quite a bit of awareness that are not apparent, and sudden shocking fear does allow us special awareness to avoid problems.  All of the statements about 'fight-or-flight' reactions are too crude to any actuality in the complexity of our reactions.  My father was a mild mannered high level civil service man in Bengal.  He was not using any of the dormant capacities but I did some times. 

\section{Rough idea of origins of rationality}

Our rationality comes with the actual adaptive problems resolved by our ancestors 75kya.  Most of these have no use in the modern world, but they do point to the source of our rationality.  We are rational certainly for survival, and then we cogitate to avoid death-provoking situations.  

Then we might have needed oral traditions to learn how to survive.  Settled agriculture begins 10kya, so much later that human nature was firmly encrusted deep in our genes. 

\section{Socrates' Propaganda about Reason}

Before the wildly successful propaganda of Socrates who corrupted the Greek youth and the entire human race with the belief that rationality is in our human nature, the idea was the most ludicrous thing in the world.  Rationality might have been at most a mnemonic to organize the teachings and mythological wisdom of Ancient Greeks but it was never part of Human Nature.  Anyone foolish enough to think through the situation of a venomous cobra in front of them is Darwinian road kill who just gave up some good opportunties to pass on their genes to progeny.  Writing is so new that the world is not even 100 percent literate today, let alone rational thought being part of Human Nature.

What is much more likely is that laziness and learning from others is part of our Human Nature when we don't want to spend grueling amounts of work on a problem someone else has solved.


\end{document}
