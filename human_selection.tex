\documentclass{article}
\usepackage{graphicx}
\graphicspath{{./}}
\title{Sexual Selection in Humans}
\author{Zulfikar Moinuddin Ahmed}
\date{\today}

\begin{document}
\maketitle

\section{Introduction}
We want to return to Charles Darwin's theory of sexual selection with the specific intention of providing a radical revision of that theory based on measured mate preferences over half a century.

We consider the explicit mate preferences to be giving us strong evolutionary information. In particular the mate preferences do give us immediate direct preference for certain personality traits: Neuroticism, Conscientiousness, and Agreeableness.  But there is a persistent high ranking of 'Mutual Attraction'.  

In this note we give a radically new model for Sexual Selection in Humans that differs from the extant models.  In particular we follow the measured mate preferences to consider Romantic Love itself along with targeting of Neuroticism, Conscientiousness and Agreeableness as extremely old, much older than the currently accepted date of pair bonding.

The structure of the argument is the following.  First we argue that since work of Helen Fisher et. al. shows that Ventral tegmental area is the locus of activity for intense romantic love, and since VTA exists in primitive form even in mice, from the very beginning of our species, approximately 300,000 years ago, pair bonding had potential, and Romantic Love was part of the evolutionary process for all 300,000 years and pair bonding has existed throughout our evolutionary history.  The rationale is quite simply that without pair bonding and parental investment of males in the early period 300,000-280,000 years ago, our species would have been extinct since human babies are quite burdensome for a single mother.  Even in the modern economy human babies are quite difficult for single mothers to handle.

\section{Physiological Machinery for Romantic Love is Ancient}

VTA is known to be part of mice and other mammals and therefore had already evolved to roughly present form 300,000 years ago. This is crucial and it is to the credit of the work of Helen Fisher et. al. that we know that VTA is fully implicated in intense feelings of romantic love.  It is unlikely that the potential for romantic love developed without pair bonding 300,000 years ago.  In addition, the cost of raising a human baby is too much for a single mother in early conditions.  Therefore we do not believe that pair bonding formed in humans 20,000 years ago with agriculture.  We believe Romantic Love was part and parcel of our sexual selection process for 300,000 years.

\section{Cultural History of Articulations of Romantic Love}

Although Plato's Symposium has a myth of souls divided in two for soul mates, the first articulate formulation of Romantic Love is by Persian Muslim philosopher Ibn-Sina (Avicenna) from 11th century.  The Crusades led to the propagation of the tenets to the West and 12th century troubadours sang about it, and in the East Jalal al-Din Rumi was prominent in 13th century for his mystical love poetry.  King Arthur Legends had begun in the 9th century but the key lover character Sir Lancelot was from 1170.  In particular the Romantic Love elements in Arthur legends were later than Avicenna.  The issue is not just precedence but a serious context for Romantic Love for many seem to believe that Romantic Love is a Western phenomenon.

\section{Buss Half Century of Mate Preferences in America}

\includegraphics{buss_pref.png}

\end{document}
