\documentclass{amsart} 
\usepackage{graphicx}
\graphicspath{{./}}
\usepackage[fontsize=14pt]{scrextend}
\usepackage{hyperref}
\usepackage{csvsimple}
\usepackage{epigraph}
\title{Negative Correlation of Rape and Robbery with Murder in 18 Country Data}
\author{Zulfikar Moinuddin Ahmed}
\date{\today}
\begin{document}
\maketitle

\section{The Correlations}

% latex table generated in R 3.6.3 by xtable 1.8-4 package
% Tue Apr  6 02:24:59 2021
\begin{table}[ht]
\centering
\begin{tabular}{rrrr}
  \hline
 & Rape & Robbery & Murder \\ 
  \hline
Rape & 1.00 & 0.21 & -0.28 \\ 
  Robbery & 0.21 & 1.00 & -0.24 \\ 
  Murder & -0.28 & -0.24 & 1.00 \\ 
   \hline
\end{tabular}
\end{table}

\section{Comments}
Murder rates are negatively correlated with rape and robbery rates in 18 country data.  This is likely a universal natural feature.  

There are other decompositions given  maternal attachment security factors that separate rape rate from both robbery and murder rate.

These ought to have universal explanations.  These results allow us to substantially improve our understanding of 'aggressive crimes' and recognize their different sources.  This is truly a remarkable advance in human nature understanding.  The failures are at least partially a function of social health of the population.  I will firmly hold the line that the genetic predisposition for crimes is extremely unlikely to have merit at all.  Crimes are expressions of failure to care for social well-being of large populations.

\end{document}
