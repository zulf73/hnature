\documentclass{amsart}
\usepackage{graphicx}
\graphicspath{{./}}
\usepackage{hyperref}
\usepackage{csvsimple}
\usepackage{longtable}
\usepackage{lscape}
\usepackage{epigraph}
\title{May 27 2021:  Zulf Proposes Entirely New Foundations of Scientific Inference} 
\author{Zulfikar Moinuddin Ahmed}
\date{\today}
\begin{document}
\maketitle

\section{Background}

Years ago when I was a wee pup, I was taking a functional analysis course at Princeton taught by the great mathematician Peter Sarnak.  I was young, and would follow along as best I could.  He spoke about the beautiful solution of Riemann for the Dirichlet problem.  The problem is $\Omega \subset \mathbf{R}^2$ is a smoothly bounded simply connected domain.  We want to find a harmonic function with prescribed boundary values.  Riemann's solution was
\[
u_0 = \mathrm{argmin}_{u \in C^2} \int_{\Omega} | \nabla u |^2 dx
\]
Karl Weierstrass hated it because he was not convinced that a minimizer will necessarily exist.  And thus began the rigorous development of all of mathematics.

I feel that I am being the Karl Weierstrass in the situation, which is not really my cup of tea.  I always admired Bernhard Riemann a great deal more than Karl Weierstrass.  

I am not being Weierstrass.  I am not suggesting that Statistical Inference is lacking a {\em mathematical} foundation.  It has many.  It is not for lack of formalisation that there is a serious problem today in Statistical Inference.  Nay, it is that we have a more substantial problem of not being clear at all of what {\em Nature} is actually.



\end{document}