\documentclass{amsart}
\usepackage{graphicx}
\graphicspath{{./}}
\usepackage{hyperref}
\usepackage{csvsimple}
\usepackage{longtable}
\usepackage{epigraph}
\title{Zulf's Theory of Profound}
\author{Zulfikar Moinuddin Ahmed}
\date{\today}
\begin{document}
\maketitle

\section{What is Profound?}

The reference for {\em profound} exists; it is not in a vacuum.  Profound refers to the fundamental problem that human mind is configured to understand certain types of things.  I won't go into what sort of things because that will side track me.  However, {\em Nature} is objective external reality.  Profound is the point at which the human mind can be assured that he has found some secret of {\em Nature}.  Secrets of Nature are never easy and there are various stratagems man has devised to attempt to discover the secrets of Nature.  

\section{Profound are the Truth of Nature Man Discovers}

Truths of Nature are not easily given to human beings; we have to struggle to decipher the laws and truths of Nature.  When we are right about truths of Nature, we have found something profound.

\section{Posturing and Writing Erudite Tomes Is Not Necessarily Profound}

Anyone can write pompous tomes that seem erudite to those who are bamboozled by posture and superficial acting.  Arthur de Gobineau's text {\em Inequality of Races} seems erudite for its pompous presentation.  But it is not profound but simply misguided.  Things that are pompous and erudite but not true are not profound.  They are wrong.  One cannot be profound without being right.  And to be right is not very easy.  

\end{document}