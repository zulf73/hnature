\documentclass{amsart}
\usepackage{graphicx}
\graphicspath{{./}}
\usepackage{hyperref}
\usepackage{csvsimple}
\usepackage{longtable}
\usepackage{lscape}
\usepackage{epigraph}
\usepackage{amsmath}
\DeclareMathOperator*{\argmax}{arg\,max}
\DeclareMathOperator*{\argmin}{arg\,min}\title{What Henry Thomas Buckle Got Wrong}
\author{Zulfikar Moinuddin Ahmed}
\date{\today}
\begin{document}
\maketitle

\section{Henry Thomas Buckle 1857}

The regularities of morals was addressed by Henry Thomas Buckle in {\em History of Civilisation in England} in 1857.  His reasoning was that regularities in crimes would imply regularities of virtues.  This part is correct.  But then he said, "if it can be demonstration that the bad actions of men vary in obedience to the changes in the surrounding society, we shall be obliged to infer  that their good actions vary in the same manner ... that such variations are due to large and general causes".

The problem here is an implicit assumption that it can be demonstrated that virtues and vices are actually influenced by external general social variables at all.  

This was a prejudice of Victorian England that did not have any significant alternative till Charles Darwin's theories would wield siginficant influence.  

Today, it is fairly easy for me to consider the dominant force behid {\em global} moral virtues and vices to deeply shared genetic inheritance from millions of years of co-evolution in Africa.  

Henry Thomas Buckle and others in 1857 did not have available inference from genetic inheritance on a global scale and therefore tended to hypothesize influence of surrounding society.  My prejudices, perhaps equally erroneous, are that uniform Exponential curves for moral convictions are evolutionary in origin and largely unperturbed by general social conditions.

\end{document}