\documentclass{amsart} 
\usepackage{graphicx}
\graphicspath{{./}}
\usepackage[fontsize=14pt]{scrextend}
\title{Greed and Money from Adam Smith to Rainer Maria Rilke}
\author{Zulfikar Moinuddin Ahmed}
\date{\today}
\begin{document}
\maketitle

\section{Context of Judgments}
On the issues of greed for Money and its relation to self-interest as well as the moral status of the greed for Money there is a vast divergence of viewpoints.  I am not an expert on Islamic and Christian viewpoints on this, so I won't mention those here; I may have inadvertant similarities to Christian or Islamic tenets because my parents were Muslim and I spent my life in America which is majority Christian, but my own intellectual path was independent.  

I will traverse the span from extreme disinterest in issues of greed for money in Rainer Maria Rilke to the Classical Liberal equilibrium position of Adam Smith, to some exaggerated paragon of the zealous believers of a religion of greed for money such as Bill Gates.

In this scale I occupy some position that is somewhere between Rilke and Adam Smith, and find Bill Gates' position to be quite deporable, abominable, unsavory, and my reaction to him is intuitive condemnation.  I do not pretend any neutrality here. I cannot stand the religion of greed for money.  It offends my sense of beauty and good.

\section{Rilke's Tenth Elegy}

Rilke's spiritual orientation was otherworldly in an extreme manner, and in Duino Elegies he shattered quite explicitly all of 'our interpreted world' and gave the description of human existence in a true reality, the 'uninterpreted absolute reality' and provided valuation of all things of existence where he places greed for money in a specific place.

In Tenth Elegy there is a carnival in  the City of Grief on the edge of town:

"For adults only there is something special to see: how money multiplies, naked, 
right there on the stage, money's genitals, nothing concealed,
the whole action--educational, and guaranteed to increase your potency"


\section{Adam Smith}

Adam Smith considers self-interest in terms of Autonomy and one's own capital as an accurate abstract model of every human being.  He was not promoting that one consider greed and expansion of wealth without any regard for any laws of justice as the be-all and end all of purpose of human existence.

Every man, so long as he does not violate the laws of justice is left perfectly free to pursue his own interest in his own way and to bring both his industry and capital into competition to those of any other man. (Wealth of Nations)

“one individual must never prefer himself so much even to any other individual, as to hurt or injure that other, in order to benefit himself, [even] though the benefit to the one [might] be much greater than the hurt or injury to the other.” (Moral Sentiment)

\section{Bill Gates}

The really easy problem is that some people love their own greed a lot and they invoke Adam Smith to justify their bottomless relentless greed.  They are avaricious beyond all limits and they don't want to care about anyone else's well-being.  They turn their own selfishness into a personal religion and then they build an altar to their own selfishness and they want Adam Smith to be their Holy Scripture.  Bill Gates is this type of a man.  His "PR" is fake.  In reality he is a greedy mofo beyond compare with absolutely no constraints.  And this is the easy problem of Adam Smith.  Adam Smith did not actually promote this money is the be all and end all and be as greedy as possible without any concern for consequences and keep repeating 'my money my money my money my precious my precccciiiioous' like the Gollum of Lord of the Rings.  That is a menial understanding of Adam Smith.  Adam Smith is not actually promoting anything so disgusting, vile, unappetizing, and abominable.  And it is an easy problem because that is just a lack of proper education about what Adam Smith considered as important.  He considered self-interest of all human beings regardless of race, religion, creed, skin colour, etc. across the world as possessing self-interest, and then considered Autonomy and Money and industry as requiring freedom.  He was much more subtle than this greed-is-not-only-good-but-divine sort of thing that some extremely greedy people do.  

\end{document}


\end{document}