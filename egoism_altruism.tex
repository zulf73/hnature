\documentclass{amsart} 
\usepackage{graphicx}
\graphicspath{{./}}
\usepackage[fontsize=14pt]{scrextend}
\usepackage{hyperref}
\usepackage{csvsimple}
\usepackage{epigraph}
\title{On Egoism and Altruism in Human Nature}
\author{Zulfikar Moinuddin Ahmed}
\date{\today}
\begin{document}
\maketitle

Years ago in 2003-4 I did some anti-war activism with International Socialist Organization.  I had a lot of time to think about what was so compelling about it for me.  And the answer is that I felt that killing all those Iraqi people and destroying their country was just wrong and I needed to soothe my Conscience that I should do something about it.  I was not about to stop the wars.  I did not know then that Israel's Mossad had staged the Twin Tower demolition remote controlling two planes into the buildings to trigger war plans already classified.  I just felt the war was wrong and I sold newspapers about Abu Ghraib torture for a buck on Mission and Valencia Streets in San Francisco.  Now you could say I did it for egoistic reasons; or you could say I did it for altruitic reasons.  In both cases you would be literally wrong.  I did it because I had to do it.  I would not feel right and my feeling about who I am in the world would have suffered greatly.  My self-respect and self-esteem would have been damaged if I did nothing at all.  And I am very proud of doing it.  It makes me a far greater man today.  Maybe no one will show me respect for it, but that does not matter to me.  I am tremendously proud that I did those things.  It was egoistic in a way and altruistic in a way and it was an expression of who I am in a way, and it does not matter how you slice it.  It was what I did, and it was Zulf doing it, and afterward Zulf was greater for having done it.  I don't see the point of considering whether to classify doing this as 'egoistic' or 'altruistic' in the end.  I don't always go out and do every possible altruistic thing for the sake of altruism.  I do things that are right and are appropriate.  This is perhaps the point, that we are naturally social mammals, and so we have biological drives and instincts that allow us to feel right about doing certain things that are pro-social.  Many philosophers like John Stuart Mill held primary doctrines such as egoism and utilitarianism that are not in accordance with our Human Nature.  Kant as well.  We do many things not for a reason at all but because that is who we are and we feel right and good and just about ourselves for doing certain things.  It seems to me that when there is a debate about Egoism versus Altruism, the first thing that goes out the window is that we are natural social mammals whose instincts were social even before we were evolved into Human Beings 300,000 years ago.  Therefore "Egoism" and "Altruism" are artificial categories for the sorts of instincts we have inherited. And this is the way to throw out all the 17th-18th century philosophers who theorised philosophies of abstract beings who do not have social instincts and selfish instincts combined deep in our genetic history and pretend that we are driven by controlled will.  Will comes in at the very end of very powerful drives.  Our social instincts are very strong but because of ARTIFICIAL categories like Egoism and Altruism we systematically keep going in circles.  These are ultimately not concepts APPROPRIATE for describing how we are the way we are or could be.   

\section{Human Nature is Neither Egoism or Altruism}

The concepts of "Altruism" and "Egoism" are both inappropriate to consider in Human Nature because we are simply social mammals.  And the poets like Rilke are right to focus on the relationship between lovers.  Now Rilke himself was deeply fond of solitary self-reflection, and yet he was profoundly developed as a lover.  As I have pointed out in an earlier paper of mine with a new evolutionary theory of romantic love, romantic love not only has neural correlates but precedes even the evolution of H. Austraulopithecus, as paternal care was necessary before big-brained bipedal primates could survive, put the age of first occurrence of romantic love before five million years ago.  I have recently argued from the maximum rape rate being 0.06\% uniformly over Earth that rapes are not Human Nature at all; they are not just crimes but there was no moment in human evolutionary history in the past several million years when rapes were ever Human Nature.  This is supported by rapists having failures in social competency mostly explained by insecure maternal attachments.  We have, I think, the wrong idea of the sort of problems that earlier ages faced.  Steven Pinker's {\em Better Angels of Our Nature} does point out that violence has declined such as murder and stateless situations would lead to Hobbesian situations.  But rape rates and some of the other crime rates would not be uniformly low across the human race unless Human Nature had discarded them. They are failure rates and I believe they are symptomatic mostly of childhood maternal attachment and adolescent paternal guidance.  Furthermore I believe that motherhood and fatherhood are both of ancient evolutionary lineage.  

Social competence is predicted by maternal attachment between 0-3.  And this is absolutely key I think to understanding our fundamental social nature.  When considering such abstract concepts such as "Egoism" and "Altruism" we are not actually considering actual human beings; rather, we are considering some abstract a who is not human.  Every human being is fundamentally affected by the {\em longest dependent relationship} among all mammals because of the time for development of the full brain and therefore we are more social beings than all other mammals in a sense.  Our relationship to our mother and father are so strong that around the world 80\% consider making their parents proud the most important goal in their life.  This fact alone should tell us that "egoism" and "altruism" are concepts alien to our nature.  Those who are insufficiently sophisticated in their social development have failed relationships, marriages, and face Darwinian extinction.  We do many social activities because we have psychological {\em needs} for them, and those require whatever mix of "egoism" and "altruism" as they do.  The extraction of some portion of our social activity had been considered "altruistic" by philosophers because they had considered "egoism" as a valid central notion for Human Nature.  Not every thing we do and think are deliberate and based on thought.  We would be completely paralyzed from inaction. We are habituated to particular social environments.  Now unselfishness we teach to our young for 26\%, and 20\% of us join charitable organizations.  The 80\% are not specifically concerned with altruism.  However, socially well-adjusted human beings do take a relatively positive view toward altruistic behaviour even when they themselves might not actually act in altruistic manner.  Around 80\% of us consider {\em work} to be duty to society. 

The above picture we get from World Values Survey tell you some features of what we are like.  It tells you that service to society is in the 80\% and altruism is in the 20\% class.  So what is happening is that the {\em concept} of altruism is not particularly central.  We are driven by social orientation and some of this is generous and altruistic and others not so much.  The alternative is not "egoism".  Self-interests are important, but social needs are not trivial.


\end{document}