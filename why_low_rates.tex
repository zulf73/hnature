\documentclass{amsart} 
\usepackage{graphicx}
\graphicspath{{./}}
\usepackage[fontsize=14pt]{scrextend}
\usepackage{hyperref}
\usepackage{csvsimple}
\usepackage{epigraph}
\title{Harvard Declared Bill Gates Deserves Death And Future Plans of Zulf}
\author{Zulfikar Moinuddin Ahmed}
\date{\today}
\begin{document}
\maketitle

\section{Why Are Rape Rates Universally Low?}

Let's take some of this highest rape rates in the world from 2013. Sweden 58.54, Iceland 55.32,  United States 35.85, Belgium 29.99, Norway 22.35.  These are representative figures.

All of these are in 100,000 per year.   The complement gives us 99.94\%.  In other words, every year assuming each of the rapes were by different men, 99.94\% of the men are not involved in a rape.  For us, it is still not high enough because we have a relative sense.  But from my point of view this is remarkably low for the human race, and that is an enigma.  How did we, as a species, manage to avoid the situation of the chimpanzees, where maybe 80\% of children are born off 'rape'?  The rather wonderful feature of human race is that 99.94\% is the least percent of non-rapists in the population, and so almost all men never rape women.  I find this regularity of rather good behaviour to be impressive because it could be Hobbesian jungle of rapes the moment the police look the other way.  But it is not.  This was the product of millions of years of evolution that our men are remarkably well-behaved with our women, and much more in the poor countries than in some of the Western wealthy countries.  Obviously 0.06\% maximum is aberrant behaviour.  And so this wonderful feature of our race requires some examination.  It is most certainly not explained by any particular religion.  For otherwise why would the regularity be globally uniform.

I want to set up the context of my universal viewpoint with this observation. 

\end{document}