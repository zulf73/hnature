\documentclass{amsart} 
\usepackage{amsmath}
\DeclareMathOperator*{\argmax}{arg\,max}
\DeclareMathOperator*{\argmin}{arg\,min}
\usepackage{graphicx}
\graphicspath{{./}}
\usepackage[fontsize=14pt]{scrextend}
\usepackage{hyperref}
\usepackage{csvsimple}
\usepackage{epigraph}
\title{Nature's Inscrutibility in Social Sciences}
\author{Zulfikar Moinuddin Ahmed}
\date{\today}

\begin{document}
\maketitle
\epigraph{Consider this, consider this

the hint of the century

Consider this the slip that brought me 

to my knees, failed

What if all these fantasies 

come flailing around?

Now I've said too much}{REM, "Losing My Religion}

\section{Social Science is more Inscrutible than Physics}

Physics has nontrivial theories and has long known the secret to Science that Social Science has struggled with for centuries.  The secret is that Nature does not give a damn about our theories.  Nature cannot even give us a contemptuous look.  The laughter of the gods might be quite upsetting for our efforts, but it's even worse with Nature, deafening silence, not even laughter and mockery for our theories.  With Physics we have some confidence from centuries of experience.  In Social Science, there is the added demand from society that our concepts make sense to ordinary laymen.  And this is then Zulf caught between Scylla and Charybdis.  Neither can you please the people who demand something they can understand, nor can you please the purests of scientists who sniffle pompously to your beautiful prediction just because there's a little bit of noise in the way.  

\section{Pure Social Science Might Not Be Comprehensible}

This is a theoretifcal experiment.  Suppose you had found the exact 50,000 factors that with their relationships give us $R^2=0.90$ statistical models no matter how you consider them.  That would be much better than anything that exists today.  That is Social Science with the sort of precision that would give an attractive supermodel an orgasm one day.  I am not so young but just trying to keep things artificially youthful; don't try that with any attractive supermodel please.  

Returning to the main point, even in such an ideal and quite far-fetched scenario, there would be a great problem.  With high likelihood, not even one of the 50,000 variables would bear any resemblance to anything that human beings naturally considered to be interesting and interpretable.

A priori, there is a conflict here between Man and Nature.  The concepts that Man finds interesting are Man's concepts; Nature has no obligation whatsoever to give any attention to any of them at all.  Nature works the way that Nature works, and is certainly not beholden to be comprehensible to Man.

This is one of the most fundamental limitations of Social Sciences.  In this consideration we are giving implicitly the sort of scenario which might allow Social Scientists to stray from the cowardly path of {\em making any sense to Man}.  I am not seeking a revolution, but pointing out that there might be no possibility of avoiding taking these steps into the unknown concepts that make no sense whatsoever for the purity of devotion to Nature.  In this dreadful Empiricist world, there are simply not enough incentive to do this in a beautiful manner.  Social Scientists could consider giving up their Empiricism, their Christianity or Islam or whatever, and begin worshipping Nature.  That is a good idea.  Without devotion and a feeling of divine calling, you will have a hard time finding motivation to seek variables that will produce a sharp Social Science.


\end{document}