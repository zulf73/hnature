\documentclass{amsart}
\usepackage{graphicx}
\graphicspath{{./}}
\usepackage{hyperref}
\usepackage{csvsimple}
\usepackage{longtable}
\usepackage{epigraph}
\title{Descriptive Statistics for Peace-loving Peoples of the World}
\author{Zulfikar Moinuddin Ahmed}
\date{\today}
\begin{document}
\maketitle

\section{Data For 49 Countries}

The following are statistics of people who have some conviction that violence is justified from 1=Never to 4 with 10=Always.  These are people who are quite reluctant to find violence justified.

% latex table generated in R 4.0.3 by xtable 1.8-4 package
% Fri Apr 30 16:07:03 2021
\begin{longtable}{rlr}
  \hline
 & Country & peace \\ 
  \hline
1 & Greece & 97.70 \\ 
  2 & Andorra & 97.10 \\ 
  3 & Germany & 96.50 \\ 
  4 & Cyprus & 96.40 \\ 
  5 & Tunisia & 96.10 \\ 
  6 & Bangladesh & 95.80 \\ 
  7 & Japan & 95.60 \\ 
  8 & Jordan & 95.20 \\ 
  9 & Egypt & 94.70 \\ 
  10 & Taiwan ROC & 94.60 \\ 
  11 & Ethiopia & 94.60 \\ 
  12 & Myanmar & 94.50 \\ 
  13 & Romania & 94.20 \\ 
  14 & Tajikistan & 93.90 \\ 
  15 & Peru & 93.80 \\ 
  16 & Nigeria & 93.60 \\ 
  17 & China & 93.50 \\ 
  18 & Lebanon & 93.30 \\ 
  19 & South Korea & 92.60 \\ 
  20 & Indonesia & 92.20 \\ 
  21 & Puerto Rico & 92.00 \\ 
  22 & Zimbabwe & 91.10 \\ 
  23 & Pakistan & 91.00 \\ 
  24 & New Zealand & 90.50 \\ 
  25 & Australia & 89.90 \\ 
  26 & Thailand & 89.30 \\ 
  27 & Colombia & 89.20 \\ 
  28 & Kyrgyzstan & 88.80 \\ 
  29 & Argentina & 88.60 \\ 
  30 & Turkey & 88.50 \\ 
  31 & Nicaragua & 87.70 \\ 
  32 & Iran & 87.60 \\ 
  33 & Brazil & 87.00 \\ 
  34 & Bolivia & 86.70 \\ 
  35 & Hong Kong SAR & 86.30 \\ 
  36 & Macau SAR & 85.90 \\ 
  37 & Ukraine & 85.00 \\ 
  38 & Ecuador & 84.60 \\ 
  39 & Russia & 83.00 \\ 
  40 & Chile & 83.00 \\ 
  41 & Kazakhstan & 82.00 \\ 
  42 & United States & 81.40 \\ 
  43 & Guatemala & 81.30 \\ 
  44 & Iraq & 80.00 \\ 
  45 & Mexico & 79.90 \\ 
  46 & Vietnam & 77.30 \\ 
  47 & Serbia & 72.70 \\ 
  48 & Malaysia & 71.10 \\ 
  49 & Philippines & 62.80 \\ 
   \hline
\hline
\end{longtable}

\section{Summary}

The mean percent for our metric of peace-loving is $\mu = 88.6\%$.  The standard deviation is $\sigma = 7.4\%$.  

Let us consider simple inferences.  First, uniformly across the world roughly 88\% of the people simply do not believe violence is a valid behaviour.  We have to make sense of this from classical philosophical conjectures.  These are people in all levels of economic development, so this is independent of wealth or poverty, of religions, ethnicity.  We can use this table to remove ethnic prejudices against Africans, against Arabs, and others in the United States.  The world is vastly peace-loving.  

\section{Importance of these data}

The analysis is trivial, but the importance is great for {\em prejudices} especially of racial type prevalent in America and Europe and elsewhere too.  The importance of this is that it is true, measured data, and people's subjective prejudices can be updated using this.  In the future, this will be standard trivial general knowledge of children.  But today, it is surprising.


\end{document}