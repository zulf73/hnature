\documentclass{amsart} 
\usepackage{graphicx}
\graphicspath{{./}}
\usepackage[fontsize=14pt]{scrextend}
\usepackage{hyperref}
\usepackage{csvsimple}
\usepackage{epigraph}
\title{Some Thoughts on Reconciliation of Science and Faith}
\author{Zulfikar Moinuddin Ahmed}
\date{\today}
\begin{document}
\maketitle

\section{History of the Science-Faith Dichotomy}

The issue of reconciliation of Science and Faith simply did not exist on Earth before 1641.  Science did not really exist before 1641.  As modern men this seems difficult to believe, but before 1641 only Faith existed.  Empiricism and Materialism and such things simply did not exist.  Descarte's Mind-Body dualism began the whole process of taking Materialism seriously, and by the time of Jean-Paul Sartre, this three-dimensional materialism became the canvas of ultimate reality.  Well it's not true.  There is a vast purely electromagnetic spatial dimension that is objectively real.  I know this by careful reasoning in Four-Sphere Theory challenging Michelson-Morley itself, from Bohm-Aharanov reality of electromagnetic potential from 1959.  And so the question takes a new form.  We have achieved a revision of the question itself.  I have proven Eternal Recurrence, re-established Fate, ousted Gravity and Beginning, Genesis of 13 billion years.  And we are back to the beginning of the question.

\section{What is Existence?}
When Rilke looked at Existence, he was quite bold and sobbed for his discomfort in the interpreted world, with firm conviction that Existence was much larger, crossed death itself and in the torrent of a river the voices of living and dead were drowned out with Angels being able to cross realms.  

I looked at Existence differently, through the lens of Mathematical Physics and produced the final laws of Macroscopic Nature, the Four-Sphere Theory.  And there is absolute certainty in my mind now, quite carefully examining the structure of macroscopic physics that while black holes are science fiction and Big Bang, there is much more objectively real than the three dimensional material world in Existence.  

The human ability to have any reliable experience of the Eternal Four-Sphere that is the background of Existence is still unclear.  What exists in this Existence is what I claim has always been the matter of what after 1641 became Faith.  

\section{Empiricism's Power}

The Power of Empiricism is proven with now 380 years, four centuries nigh of experience and not much is needed to be said more in its favour.  However what does need emphasis is that Empiricism is not enough, not enough now and not enough even in the distant future when all things that can be established by Empiricism has been established. There are variety of reasons for this: the complement of the physical world in Four-Sphere is not going to be reachable except by indirect inferences.  The regularity laws of Nature are insufficient to ensure knowledge by pure inference even from a gigantic Borgesian Library of data and inference.  Scientific knowledge might not be valuable in knowledge necessary to live in the world.  The clear conclusion is Empiricism is necessary for Truth but insufficient.

\section{What is Reliable besides Empiricism}

Now Faith as we call the sort of thing that before 1641 was just collective reality of cultures and societies, or 'religion' which is really an academic term because the faithful have faith in their Spiritual Truth and so do not consider it in their own lives to be religion.  Today of course we are in Classical Liberal Hegemony and suddenly everyone is "Spiritual but not Religious" and faith is driven underground into the deep recesses of the subjective as a private matter.  It's true that vast majority of Earth are religious still but the legitimate governance and laws around the world are Classical Liberal.  

I think the question of what else is reliable besides Empiricism is completely unresolved and the question was taken up by some of my heroes early in twentieth century most clearly by Robert Musil.  Romantic Moderns, poets and writers, Carl Jung, and there is no clarity in sight.  Now Nietzsche was quite interested in some of these issues and he considered the issues to various sorts of breakdowns.  

I think that a collective appreciation for some issues of Faith occurs Naturally because we're a social species and cannot tolerate isolation for too long, and we naturally gravitate towards shared faiths.  I don't mean the formal ones but the beliefs that catch on and just become background reality in our minds that we accept either out of conviction or resignation.

\end{document}
