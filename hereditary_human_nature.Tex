\documentclass{amsart}
\usepackage{graphicx}
\graphicspath{{./}}
\usepackage{hyperref}
\usepackage{csvsimple}
\usepackage{longtable}
\usepackage{lscape}
\usepackage{epigraph}
\title{Unbearable Lightness of Hereditary} 
\author{Zulfikar Moinuddin Ahmed}
\date{\today}
\begin{document}
\maketitle

\section{When Your Assessment of Hereditary Is Misguided}

Suppose you are measuring a trait of a human being $T$.  And suppose you do a tests on twins and such and you find that 70\% of trait $T$ is hereditary.  Can you then just say well, it's hereditary so we ought to pay enormous attention to the parents and the the ancestors etc.

I will now tell you how this viewpoint which is quite common in psychology can be completely misguided.  

At first glance, this seems wrong.  How can hereditary traits not be fundamentally about parents and ancestors?

I will propose a different viewpoint that is more sensible in many of these cases.  The viewpoint is based on Human Nature.  Certain traits $T$ which are measured and found to be hereditary are actually common to all humans and the hereditary aspect is deep in human evolutionary history going back more than 75,000 BP and 2 million years and it is therefore a bad judgment to take a {\em local view} for such traits because Human Nature traits will seem hereditary on some studies without giving us insight about these traits at the macroscopic Human Race level where they show their true sense.

I will claim that Life Satisfaction, is hereditary and also Human Nature at the same time, and the {\em local heredity} of Life Satisfaction and other traits of similar class will deceive us from understanding them.

\section{Life Satisfaction By Ethnicity}

% latex table generated in R 4.0.3 by xtable 1.8-4 package
% Wed Jun  9 20:36:53 2021
\begin{table}[ht]
\centering
\begin{tabular}{rr}
  \hline
 & satisfied \\ 
  \hline
Black & 74.51 \\ 
  East Asian & 94.37 \\ 
  Indian & 95.63 \\ 
  Other & 86.05 \\ 
  White & 89.64 \\ 
   \hline
\end{tabular}
\end{table}

These are simple summary of Life Satisfaction 6-10 by ethnicity across the globe.  The standard deviation of these numbers is $\sigma=8.47$.

Now many studies have shown that Life Satisfaction has heredity from 50-70\% \cite{BPD}.  At the same time our result here tells us that across ethnicitie across the globe the genetic diversity included, Life Satisfaction is constrained to 8.5\% of variability.  Here the focus on the local is misguided because we are dealing with deep inheritances that go back to our evolutionary history and individual variation is actually a small perturbation of the macroscopic value.  

The perspective is not a minor issue.  It is a major shift in how we see these traits.  A priori, genetic inheritance could even be totally unique to a small family, and this is where there is a misguided sense in over-emphasis on percentage heritability in psychology.  There are many traits of this type that require the balanced perspective of macroscopic {\em bounds of variablity} and microscopic precision of individual heritability.


\begin{thebibliography}{CCCC}
\bibitem{BPD}{Braungart,  J.  M.,  Plomin,  R.,  DeFries,  J.  C.,  and Fulker, D. W. (1992). Genetic influence on tester-rated infant temperament as assessed by Bayley’s Infant  Behavior  Record:  Nonadoptive  and  adoptive  siblings  and  twins.Developmental Psychology, 28,40 – 47.}
\end{thebibliography}
\end{document}