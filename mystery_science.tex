\documentclass{amsart}
\usepackage{graphicx}
\graphicspath{{./}}
\usepackage{hyperref}
\usepackage{csvsimple}
\usepackage{longtable}
\usepackage{lscape}
\usepackage{epigraph}
\title{The Fundamental Mystery of Science}
\author{Zulfikar Moinuddin Ahmed}
\date{\today}
\begin{document}
\maketitle

\section{Human Mind Did Not Create Nature}

In the Axial Age, all the great religious traditions of Man developed and the history is fascinating.  The quest for understanding existence began, and gods were invented of great powers, and human civilisation grew and developed.  I love Joseph Campbell's idea that in every age and every place Man had been animated by mythology.  The Scientific Revolution with Rene Descartes and Francis Bacon is relatively recent; Descartes lived 1596-1650 and Francis Bacon 1561-1626.  Not even five centuries have been the founding of modern empirical scientific view.  There is in empiricism a great optimism, one that does provide us with progress in deeper understanding of Nature.  But at the same time, the fundamental question of Man's ability to decipher Nature is not resolved.  We can celebrate any theory of Nature we like, believing that Man has indeed triumphed and the laws of all things of Heaven and Earth have been elucidated.  My own personal view is that this is always folly, for simply our great confidence that we have indeed captured understanding of all of Nature is no great solace to anyone who considers the history of errors that the entire human race has traversed.  

The optimism of Science must always be kept in check lest we fall into one of our most enduring traits.  Friedrich Nietzche had considered human beings are peculiar in his continual need to believe in delusional tales for survival perhaps.  And he wrote {\em The Birth of Tragedy from the Spirit of Music} as a critique of science from an artistic metaphysics.  It is a marvelous book, and quite profound.

Nature is a complex and variegated phenomena.  Our perception of what is the case is imperfect, and so the mystery of Nature will remain with us for millenia.  

There is always, in my mind, the question of whether our best theories of Existence, of Nature, actually are reading her secrets or whether we have found new delusions to replace older ones.  

There are no laws of Nature that I, at least, know about that says that we will have any entitlement to knowledge of the actual secrets of Nature.  Our concepts are our own, and Nature will cast off our concepts with mirth and amusement without revealing any of her actual secrets regularly.  

\section{It is Always Extremely Serious When any Mathematical Model of Nature Produces Predictions}

We believe that our mathematical models of Nature are actually those of Nature, but there is no assurance at all that this is the case, no certainty.  Thus every mathematical model of Nature that produces any predictions that can be verified is a profound miracle.  





\end{document}