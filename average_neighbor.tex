\documentclass{amsart}
\usepackage{graphicx}
\graphicspath{{./}}
\usepackage{hyperref}
\usepackage{csvsimple}
\usepackage{longtable}
\usepackage{lscape}
\usepackage{epigraph}
\title{Human Race Average Neighbor} 
\author{Zulfikar Moinuddin Ahmed}
\date{\today}
\begin{document}
\maketitle


\section{The Average Neighbor}

% latex table generated in R 4.0.3 by xtable 1.8-4 package
% Tue Jun  1 13:08:40 2021
\begin{table}[ht]
\centering
\begin{tabular}{rlr}
  \hline
 & trait & p \\ 
  \hline
1 & drug addicts & 85.02 \\ 
  2 & diff race & 12.82 \\ 
  3 & have AIDS & 38.96 \\ 
  4 & foreign workers & 20.13 \\ 
  5 & homosexuals & 47.29 \\ 
  6 & diff religion & 12.71 \\ 
  7 & heavy drinkers & 63.71 \\ 
  8 & unmarried couples & 23.07 \\ 
  9 & diff language & 13.51 \\ 
   \hline
\end{tabular}
\end{table}

This is quite interesting to me because it is impossible for me to guess these things and measurements are truly key to estimating these probabilities.  

What immediately occurs as an interesting issue is that different race, religion, and language are around 12-14\% around the world.

Then drug addicts are 85\% and heavy drinkers 64\%.  Having AIDS is 40\%, homosexuals are 47\% and unmarried couples 23\%.

\section{Spot Check:  Difference and Tolerance Values}

For those who do not have neighbors of different race, religion, or language, the child rearing quality of tolerance and respect for others is 63\%, while those who do have such neighbors, it is lower by around 2-4\%.

\begin{verbatim}
> table(na.omit(polv[,c("Q12","Q19")]))
   Q19
Q12     1     2
  1  1339 10203
  2  1012  5791
> 10203/(10203+5791)
[1] 0.6379267
> 1339/(1339+1012)
[1] 0.5695449
> table(na.omit(polv[,c("Q12","Q23")]))
   Q23
Q12     1     2
  1  1395 10147
  2   936  5867
> 10147/(10147+5867)
[1] 0.6336331
> 1395/(1395+936)
[1] 0.5984556
> table(na.omit(polv[,c("Q12","Q26")]))
   Q26
Q12     1     2
  1  1529 10013
  2   949  5854
> 10013/(10013+5854)
[1] 0.6310582
> 1529/(1529+949)
[1] 0.6170299
\end{verbatim} 
This is intriguing, so existence of difference lowers the enthusiasm for tolerance and respect.

\section{The Problem With Bill Gates Believing When He Sees It}

Bill Gates is a high school educated man and so his mind is too primitive to grasp the idea that some features of nature cannot actually be seen by normal perception of human beings.  In physics, this is quite clear that a man who says "I will believe in an electron when I see it is deluded, since electrons are far too small for normal eyesight.  

But for some reason, the primitive scientifically illiterate mind believes it capable of seeing subtle issues of social phenomena that require vast numbers of samples.  This is not possible to "see" either with ordinary senses.  Bill Gates, who will be spending time with his harem of girls who are quite mature for their age in his yacht is not going to be able to "see" any social phenomena that requires diligence and interviews of tens of thousands of people.   That is not possible.  He does not have sufficient abilities to "see" these things.

The important point is that measurements of tens of thousands of people is a far more reliable path to objective knowledge than completely random considers and prejudices and supersitious convictions without any basis in reality, which is what drives the isolated narcissistic minds of people like Bill Gates.

\section{White People Increase in Tolerance with Religious and Linguistic Diversity, Reduce for Racial}

First of all the effects are quite small.  I want to focus on White People here because Bill Gates is extremely barbaric in his racial stance and is genocidal in his hatred against non-whites.  Most white people are not like this.  I will construct a little table.

% latex table generated in R 4.0.3 by xtable 1.8-4 package
% Tue Jun  1 14:56:00 2021
\begin{table}[ht]
\centering
\begin{tabular}{rrr}
  \hline
 & nondiverse & diverse \\ 
  \hline
neighbors other race & 62.80 & 58.00 \\ 
  neighbors other religion & 61.34 & 64.60 \\ 
  neighbors other language & 61.56 & 64.43 \\ 
   \hline
\end{tabular}
\end{table}
The interpretation is the following.  These are percentages who think it is important to raise kids with tolerance and respect for others. Note that the lowest number here is 58\%.  White people are quite committed generally to raising their children with tolerance and respect for others.  Bill Gates is not just exceptional in his views but he's a total freak among white people regarding racial supremacy and other delusional nonsense.  His values are just far outside the bounds of white norms.

Above we see enthusiasm for tolerance and respect for others {\em increase} in the latter two rows for White people.

I will return to the corresponding statistics for other ethnicities later on.

\section{Zulf's Reaction to Bill Gates Will That Tolerance Does not Take Hold In America}

Bill Gates' knowledge of American history is so deluded that he thinks that America {\em won} the Vietnam War.  Tolerance goes back to John Locke, and Civil Rights Movement of the 1960s continued it and reduced intolerance and this is before I was born.  Bill Gates was apparently not aware of this because he lived in la la land exploiting technology workers and cutting up their eyes and stiffing them of salaries to make money.  I am quite happy to see ordinary tolerance rise among the human race and reserve my extreme intolerance for Bill Gates, who ought to be vaporised into cosmic dust and erased from History altogether with extreme prejudice.  These sorts of cunts ought to be aborted before birth.

\section{Zulf Lays Down Some Normative Principles}

My principles are ethnicity-independent.  I respect people who respect me and disrespect people who disrespect me.  I don't care about your ethnicity at all.  I don't give anyone Indian privilege or white privilege or black privilege or any other ethnic hocus.  I don't care.  If you want my respect, you have to respect me.  If you don't, you can kiss any chance of respect from Zulf goodbye.  I advocate people everywhere adopt this normative principle for themselves.  Never ever respect people who denigrate you; it is bad policy.

\section{All Ethnicities Tolerance Effects By Diversity of Neighbors}

% latex table generated in R 4.0.3 by xtable 1.8-4 package
% Tue Jun  1 15:56:47 2021
\begin{table}[ht]
\centering
\begin{tabular}{rrrrrrr}
  \hline
 & race.nondiv & race.div & relig.nondiv & relig.div & language.nondiv & language.div \\ 
  \hline
Arab & 0.80 & 0.89 & 0.79 & 0.89 & 0.79 & 0.90 \\ 
  Black & 0.72 & 0.57 & 0.71 & 0.70 & 0.71 & 0.60 \\ 
  East Asian & 0.54 & 0.42 & 0.54 & 0.38 & 0.54 & 0.44 \\ 
  Indian & 0.74 & 0.75 & 0.75 & 0.70 & 0.74 & 0.75 \\ 
  Other & 0.65 & 0.56 & 0.65 & 0.57 & 0.65 & 0.60 \\ 
  White & 0.63 & 0.58 & 0.61 & 0.65 & 0.62 & 0.64 \\ 
   \hline
\end{tabular}
\end{table}

\section{Very Subtle Significant Increase in Tolerance based on Income}

I take Q12 factored over Q288 relative wealth level.

\begin{verbatim}
> summary(lm(inc.tol~t10))

Call:
lm(formula = inc.tol ~ t10)

Residuals:
       Min         1Q     Median         3Q        Max 
-0.0143656 -0.0054485  0.0009772  0.0023195  0.0148660 

Coefficients:
             Estimate Std. Error t value Pr(>|t|)    
(Intercept) 0.6126921  0.0061883  99.008 1.21e-13 ***
t10         0.0032987  0.0009973   3.307   0.0107 *  
---
Signif. codes:  
0 ‘***’ 0.001 ‘**’ 0.01 ‘*’ 0.05 ‘.’ 0.1 ‘ ’ 1

Residual standard error: 0.009059 on 8 degrees of freedom
Multiple R-squared:  0.5776,	Adjusted R-squared:  0.5248 
F-statistic: 10.94 on 1 and 8 DF,  p-value: 0.01074
\end{verbatim}

This is a nice subtle result.  Certainly this is saying that Nature considers significant digits here important; the difference between 0.6 and 0.63 is important for tolerance level.





\end{document}
