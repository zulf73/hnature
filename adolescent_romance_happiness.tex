\documentclass{amsart} 
\usepackage{graphicx}
\usepackage{tabularx}
\graphicspath{{./}}
\usepackage[fontsize=14pt]{scrextend}
\usepackage{hyperref}
\title{Adolescent Love Changes Happiness by Small Perturbation}
\author{Zulfikar Moinuddin Ahmed}
\date{\today}
\begin{document}
\maketitle
I absolutely adore the issues of Romantic Love.  I do believe that Romantic Love is very important.  But here I report a simple result, the effect of happiness of having a romantic partner in the past 18 months among adolescents.  

\section{Baseline Result For Effect of Romantic Partner on Happiness}

We all have multiple views about effect on happiness of romantic love.  The result here is to establish a sober measured baseline for our intuitions.  For adolescents, there is a small effect on actual happiness of having had a romantic partner in the past 18 months, statistically.  

\section{Happiness with and without romantic partner}

Data are from \cite{AH}.

\begin{center}
\begin{tabular} { | c | c | c | c | c |}
 \hline
  partner & rarely & sometimes & a lot & most of times \\
 \hline
 single & 3.14 & 18.86 & 39.15 & 38.86  \\
\hline
 couple & 2.2 & 18.16 & 43.0 & 36.6  \\
\hline
\end{tabular}
\end{center}

\section{Implications}

If you are adolescent and you don't have a romantic partner, it won't change your happiness level very much if you were to start a relationship according to the small perturbation.  You might be missing out on life experiences but you are no less happy than those in a relationship.  This is the more interesting side for you might feel that you are missing out on something important, since all the movies and the social pressures seem focused on it.  In fact it is not a serious issue at all and around 45 percent of adolescents are single.

On the other hand, if you do decide to enter into a romantic relationship, you will have a mild effect on your happiness.

\begin{thebibliography}{CCC}
\bibitem{AH}{\url{https://dataverse.unc.edu/dataverse/addhealth}}
\end{thebibliography}
\end{document}
