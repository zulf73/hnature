\documentclass{amsart}
\usepackage{graphicx}
\graphicspath{{./}}
\usepackage{hyperref}
\usepackage{csvsimple}
\usepackage{longtable}
\usepackage{lscape}
\usepackage{epigraph}
\title{Zulf is Just Astounded that Probability Theory Matters At All}
\author{Zulfikar Moinuddin Ahmed}
\date{\today}
\begin{document}
\maketitle

\section{In High School I Never Liked Probability Theory}

I had to do some probability in high school.  All sorts of selecting cards from a deck, flipping coins, etc.  I hated it.  I mean I could do it because I was capable, but I did not really like it.  I was more fond of topology and linear algebra, and calculus, algebra.  I thought those were matters of consequence, the real mathematics.  I took courses in Queens College with James Munkres book on Topology.  Summers I spent in Ohio doing number theory.  What I did not realise is that I had a deep-seated prejudice against probability theory.  I just did not pay attention enough to this.  I did not like Quantum Theory at all at Princeton.  I did not take any physics courses because I read Von Neumann's book.  The Analytical infrastructure was okay, but all this randomness of Quantum Mechanics I did not like.  Later I read Einstein-Podolsky-Rosen and issues of Spin etc.  I resolved the Spin issues without problem.  It's not a feature of particles but of the universe, a spin bundle on a four-sphere.  When I considered that I knew all the randomness was wrong.  Einstein was wrong about independence of particles but he was right about God not playing dice with the Universe.  His instincts were right there.  So I was on a campaign to return to a classical deterministic physics and I succeeded some years ago.  I felt that I had rescued the Human Race from  folly.  I felt right, purely deterministic universe, so nice, so satisfying, finally.

I remember at Princeton I did not take a single probability theory course.  The thought never occurred to me that it might be a good idea.  I was committed to smooth this and analytic that, and thought of probability calculations as the sort of things that a different class of people do, people who are good at Putnam and all sorts of fast problems, and are insufferable with their glib computational mind.  I was deeper, more sophisticated, I thought.

I had to do some stochastic analysis of course after college, at Lehman Brothers.  It was a job, so it was okay.  I thought of it as things that one does when one is responsible adult.  But deep down I thought the whole randomness was shabby affair, Man's imperfections in the face of lack of knowledge, not something that was really deep.  

I even have some great stochastic volatility models.  For me it just did not occur to me that they might actually be describing anything fundamental at all.  I thought, "Well, it's finance, not physics, so it's okay.  The whole field is standing on thin ice anyway because there are no deductions and there's arbitrage in medium frequency so what's with all the martingale measures anyway?"

Don't get me wrong, I can understand martingales and Markov processes etc.  But I did not think that they were actually ever going to describe intrinsic Nature.  I thought they were always going to stay shabby stochastic approximations of things we don't want to dig into.  "Ah, well everything else is noise except whatever we think is interesting, just put an plus epsilon and draw it from distribution or other.  We don't know what's in it."  I thought that was the role of probability theory in Science.  I am quite serious.  I thought probability theory was important because we were going to lazy our way to the end of time, and will always have a need to wave our hands dismissively at all sorts of things in scientific models dismissively and call it noise.  Now Dan Stroock and others spent their lives on the subject, so I did not bring up these sorts of issues with them.

I thought that with Four-Sphere Theory, perfection of Nature would shine through and Determinism would take hold.  Why is there randomness?  Well, matter particles are in a complicated messy distribution, and we don't know what is where, so we have this gigantic "rug" discipline whose importance is that we don't like something and we just put it into stochastic noise.  

\section{Shocking Realization of May 2021}

Well the Markov Chain model works for Morals.  Did I design things this way?  No.  I had nothing to do with Human Nature Morals.  I would have designed it more with Nobility and Chivalry in the hearts of men and so on.  But I had nothing to do with Human Nature Morals.  It just works in this stochastic manner.  It's my theory so I should be quite fond of this.  Well I am quite horrified, at least at the moment.  I did not think stochastic processes would ever actually describe anything but just noise.  And now Human Moral Nature seems to be governed by actual stochastic processes.  And now I am looking at highly nontrivial Barndorff-Nielsen (Generalised Hyperbolic) distributions in all of Man's Preferences and values.  And I am sure that Levy Process Theory will tell us more.

I want everyone to know that I did not design all this.  I had no idea that stochasticity actually even mattered.  I though stochastic theory exists so that we can do smooth science and get rid of uncertainty using probability theory.



\end{document}