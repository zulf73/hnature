\documentclass{amsart} 
\usepackage{amsmath}
\usepackage{verse}
\DeclareMathOperator*{\argmax}{arg\,max}
\DeclareMathOperator*{\argmin}{arg\,min}
\usepackage{graphicx}
\graphicspath{{./}}
\usepackage[fontsize=14pt]{scrextend}
\usepackage{hyperref}
\usepackage{csvsimple}
\usepackage{epigraph}
\title{Around the World Respect for Authority increases Unhappiness}
\author{Zulfikar Moinuddin Ahmed}
\date{\today}
\begin{document}
\maketitle

Our argument is simple inference from global data.  We find correlations:

\[
\rho( AuthResp, Autonomy ) = -0.58
\]

Also

\[
\rho( AuthResp, trustMost ) = -0.395
\]

Both Autonomy and trustMost predict Happiness, ergo AuthResp is anticorrelated with Happiness.

\section{Reminder of My Happiness Model}

\begin{verbatim}
> summary(mod.xx)

Call:
lm(formula = happy ~ finsat + lessWork + trustMost + Autonomy, 
    data = xx)

Residuals:
    Min      1Q  Median      3Q     Max 
-21.375  -3.212   0.524   4.184  11.229 

Coefficients:
            Estimate Std. Error t value
(Intercept) 56.27512    3.58631  15.692
finsat       0.37943    0.05760   6.587
lessWork     0.09572    0.04344   2.203
trustMost    0.17174    0.05510   3.117
Autonomy    -0.20675    0.11292  -1.831
            Pr(>|t|)    
(Intercept)  < 2e-16 ***
finsat      5.73e-09 ***
lessWork      0.0307 *  
trustMost     0.0026 ** 
Autonomy      0.0711 .  
---
Signif. codes:  
0 ‘***’ 0.001 ‘**’ 0.01 ‘*’ 0.05 ‘.’ 0.1 ‘ ’ 1

Residual standard error: 5.708 on 74 degrees of freedom
Multiple R-squared:  0.5321,	Adjusted R-squared:  0.5068 
F-statistic: 21.03 on 4 and 74 DF,  p-value: 1.297e-11
\end{verbatim}

\section{Comments}
Our metric AuthRespect is the percentage of people who think higher respect for authority is a good thing for the future.  Now it might be that people who are unhappier think things might improve if more people had respect for authority.  This is wrong.  Trust for {\em most people} and respect for most people is better correlated with collective happiness.  

\end{document}