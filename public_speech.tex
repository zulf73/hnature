\documentclass{amsart} 
\usepackage{graphicx}
\graphicspath{{./}}
\usepackage[fontsize=14pt]{scrextend}
\usepackage{hyperref}
\usepackage{csvsimple}
\usepackage{epigraph}
\title{A Theory of Public Communication}
\author{Zulfikar Moinuddin Ahmed}
\date{\today}
\begin{document}
\maketitle
\epigraph{Did they get you to trade

Your heroes for ghosts?

Hot ashes for trees?

Hot air for a cool breeze?

Cold comfort for change?

Did you exchange

A walk-on part in the war

For a leading role in a cage?}{Pink Floyd, "Wish You Were Here"}

\section{Two Features of Public Communication}

Public communication is about two things:
\begin{itemize}
\item Values that are implicit in the message
\item An implied promise to the public that you will champion and uphold those values
\end{itemize}

\section{Role of Rhetoric}

I do not believe that many theories of rhetoric are valid personally, including that of Edward Bernays about Propaganda.  I also do not believe that Persuasion is important.

Instead, I believe that Public Communication is simply about transmission of your values -- regardless of the content--and an implicit promise that you will uphold and champion those values.

The public will have a distribution of interest; the public may or may not be persuaded; but as we let $n\uparrow\infty$ for the public, the only thing that is transmitted are values and all other things are integrated out.

\section{The Fallacy of Persuasion}

I believe that it is wrong to think about persuading anyone of anything at all.  If anyone actually receives your public communication, they will either (a) be interested in reading or hearing it, and then (b) process the values by their intuitive fast capacity to judge your values by their own moral judgment apparatus.  Then they will have their reaction.  Their reaction is unknown and is not worth attempting to model.  Their reaction should never be your concern.  You should promote your values.

\section{Inference:  Always communicate only your values}
Your actual values that you hold with conviction are worth communicating but nothing else.  This is because attempts to persuade and influence are actually folly.  They are a giant waste of precious moments in your life.

\section{Cost of public communication of wrong values is high}
The cost to your own psyche to propagate anything but your actual values is immeasurable.  And even if effective in influence, the effectiveness will be costly to maintain and will not have lasting effect because you will not stand behind them as they have no concviction and so you will lose the respect of the public eventually.

\section{It is impossible to move values by imposition of power}

This is peripheral to the main argument in this note.  There are some who believe that imposition by power or coercion will move people against their own convictions.  This is folly.  This is impossible and this was clear to John Locke already and even before.  Indeed Christianity particularly developed precisely to be resistant to imposition of power by Rome.  The processes of value formation in individuals and effects are a complex topic.  What is clear is that for public communication, these tactics are inefficient and ineffective.  Public communication is signal of values and the recipient can be influenced or not.  The key point of the note is that it is not worthwhile to attempt to influence anyone at all.  Either your values have natural capability to influence them or they will not. 


\end{document}
