\documentclass{amsart}
\title{Phase Transition in Trust}
\author{Zulfikar Moinuddin Ahmed}
\date{\today}
\begin{document}
\maketitle

\section{Introduction}
The greatest difficulty of psychology as a science is to overcome our own prejudices.  We all have our own notions about human nature, and we need them in order to function.  The most difficult task is then to separate 
our own notions about human nature, which have a depth but is {\em not science}.  The discipline required in psychology is to keep continuous focus on our own assumptions about others and continuously re-test them and improve them when careful measurements are available.  

The topic of this note is an observation about human nature as it exists external to ourselves, when we look at a feature that is interesting and surprising in data.  

\section{My Intuition} 
My own intuition was since I was a young boy was that human beings are roughly rational, and Daniel Kahneman has earned a Nobel Prize in Economics for examining this.  I am familiar with some of the irrationalities that we exhibit.

The issue of particular concern is {\em trust}.  A simple-minded intuitive model that is consistent with social penetration theory is that we have layers of onions, and thus we restrict some topics from communication with a romantic partner and talk freely about others.

\section{No Difference Between Limited and Unlimited Trust in Romantic Partners}

In the dataset we consider, ECR 1 March 2018, we will consider the difference in limited trust in communication and unlimited trust.  The surprise here is that there is no difference in the case of romantic partners between limited and unlimited trust.


% latex table generated in R 3.6.3 by xtable 1.8-4 package
% Sat Dec  5 01:49:41 2020
\begin{table}[ht]
\centering
\begin{tabular}{rrrrrr}
  \hline
 & 1 & 2 & 3 & 4 & 5 \\ 
  \hline
1 & 3.69 & 0.78 & 0.25 & 0.19 & 0.16 \\ 
  2 & 3.41 & 7.35 & 1.81 & 1.41 & 0.31 \\ 
  3 & 1.50 & 5.17 & 4.71 & 2.75 & 0.47 \\ 
  4 & 1.31 & 7.06 & 7.96 & 22.61 & 4.80 \\ 
  5 & 0.29 & 0.75 & 0.97 & 6.29 & 14.00 \\ 
   \hline
\end{tabular}
\end{table}
This is the joint frequency of (a) I discuss problems and concerns in rows and (b) I talk just about anything.

We can read off this table that 
\[
P(A < 4 | B = 5) = (0.29 + 0.75 + 0.97)/100.0
\]  
I like to make these simple calculations clear because it allows us to examine the issues involved.  It is roughly 2 percent, miniscule.

Similarly, we see 
\[
P( B < 4 | A=5) = (0.16 + 0.31 + 0.47)/100
\]
This is quite remarkable for psychology for it tells us that for romantic relationships, a trust for discussion of problems and concerns is the same as full trust for all things.

This is worth investigating in more depth, for trust levels are not considered often as having a phase transition.  Moreover trust is crucial for romantic relationships.


\end{document}