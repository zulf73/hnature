\documentclass{amsart} 
\usepackage{graphicx}
\graphicspath{{./}}
\usepackage[fontsize=14pt]{scrextend}
\usepackage{hyperref}
\usepackage{csvsimple}
\usepackage{epigraph}
\title{Moral Laws Are Not Determined by Nature but Require Habituation}
\author{Zulfikar Moinuddin Ahmed}
\date{\today}
\begin{document}
\maketitle

\section{Jonathan Haidt's Perspective On Virtues}

Jonathan Haidt's book {\em The Happiness Hypothesis} in Chapter 8, "The Felicity of Virtue" traverses through history.  He finds in Ancient Greeks the birth and seeds of later failure of virtue ethics.  Virtue ethics, he points out are not based on a parsimony principle or on Reason.  Greek introduction of principles of science then triumphed in eighteenth century Europe, especially Immanuel Kant, and this sabotaged Virtue Ethics. 

\section{Return to Aristotle phronesis}

I was the first in millenia to point out that phronesis is not a cognitive enterprise.  Plato and his successors held that reason is a gift of the gods, and separates human beings from animals.  I think that Reason as a part of Human Nature is simply not correct.  Pure Reason is a technical matter, while moral judgment is different; it is based on habituation and intuitive.

Jonathan Haidt and Joshua Greene's 2001 model of intuitive moral judgment then allows us to return to habituation as the source of moral virtues.  I will not pretend to be objective here.  Virtue Ethics is correct and deontology and other Reason-based ethics make a strong {\em ideological error} that human mind is actually Reason-based.  This is the propaganda of Socrates that held sway. It is his theory, and it is not correct.  Human Nature does not contain Reason.  Now there are many good attributes of Reason, and Reason has done a great job for modern civilization, but it is not Human Nature.

Human beings are quite capable of many aspects of Pure Reason in varying degrees of competence.  But we need to habituate to reasoning and we need to habituate to virtuous conduct, and without habituation, we do not have automatic access to Pure Reason or other virtues.  This is quite clear to me from a life of 48 years from experience.  

\section{Human Beings are Unable to Follow Dictates of Reason}

It is a simple idea that we could believe and take on faith that human beings are sentient self-aware beings endowed with reason.  But this is not actually true.  Without habituation and practice we are not able to follow any deduction from any system of values with any fidelity at all. Thus while the idea that Reason is part of Human Nature is attractive, it is simply not actually the case with actual human beings {\em ever}.  There are no people on Earth who have Reason as part of their Human Nature.  Reason is a specialty for experts who have been practicing Reason for most of their lives; it is not the natural equipment of any human being at all.

\section{Why Would We Want to Assume that Human Beings are not Reason-based?}

It is quite nice to imagine a species whose Reasoning abilities are extremely good.  However, human beings are not such a species, and we will produce a disaster for the well-being of the Human Race if we attempt to put the Human Race into various systems which assumes that Human Race is Reason-bound.  We could have views that practice and habituation of Reasoning for young children is a good thing for them, but we will be making a devastating error if we believe that Reason will be part of anyone's natural habits without enormous effort.  

\section{Only Some Basic Reasoning is Essential}

To make this issue clearer, we can consider Emotional Intelligence as the sort of thing that is more important for people than Reason.  I am not an expert but there are many studies that implicate Emotional Intelligence to success in careers in the modern world.

\section{Reason is Valuable to construct Institutions}

Regardless of whether Reason is part of Human Nature, it is valuable for construction of Institutions such as Nation States.  

\section{What Separates Us From Animals Then?}

Throughout history we have been refiners of mythologies where we record and preserve exemplary virtues and conserved Aesthetic elements.  Reason can be placed as another mythology, and like others it has positive influence on Civilization and in some senses Reason and ideals of Truth, Liberty, and Love are attached to mythologies of a modern moment.  But it is still an error to consider Reason a part of Human Nature.  This is equivalent to Judeo-Christian or Islamic mythology of the creation of Man with Divine Essence of God.  These are extraordinary hopes rather than precise truth.

\section{Virtue Ethics is Obviously True}
At a very abstract level virtue ethics is self-evident.  It says you will act in the manner to which you are habituated.  I will return to more precise ideas about Virtues another day.

\end{document}