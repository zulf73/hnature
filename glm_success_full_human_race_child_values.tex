\documentclass{amsart}
\usepackage{graphicx}
\graphicspath{{./}}
\usepackage{hyperref}
\usepackage{csvsimple}
\usepackage{longtable}
\usepackage{lscape}
\usepackage{epigraph}
\title{GLM Success for Child Rearing Values For Full Human Race} 
\author{Zulfikar Moinuddin Ahmed}
\date{\today}
\begin{document}
\maketitle

\section{Poisson Regression Succeeded}

After work with Binomial Regression, we found Poisson Regression provided best results.

\section{GLM Fits Successful with Logit Link for Full Human Race}

The full Human Race data have $N=19077$ rows and after a day of struggle I can get good fits.

\section{Do Not Have Great $p$-values for GLM Fit For Ethnicities}

I suspect the issues is sample size because the fit looks fine on "Other" ethnicity with larger sample size.

Let's take a look at the $p$-values I obtain now (for the coefficient $a$).

\[
logit(p) = ax + b
\]

% latex table generated in R 4.0.3 by xtable 1.8-4 package
% Mon May 24 01:17:15 2021
\begin{table}[ht]
\centering
\begin{tabular}{rrrrrrr}
  \hline
 & Arab & Black & East Asian & Indian & Other & White \\ 
  \hline
1 & 1.00 & 0.98 & 0.91 & 0.99 & 0.44 & 0.67 \\ 
  2 & 0.91 & 0.92 & 0.67 & 0.96 & 0.49 & 0.64 \\ 
  3 & 0.95 & 0.89 & 0.85 & 0.98 & 0.47 & 0.69 \\ 
  4 & 0.93 & 0.92 & 0.72 & 0.96 & 0.53 & 0.68 \\ 
  5 & 0.93 & 0.90 & 0.87 & 0.95 & 0.51 & 0.65 \\ 
  6 & 0.96 & 0.96 & 0.83 & 0.97 & 0.43 & 0.67 \\ 
  7 & 0.97 & 0.93 & 0.68 & 0.95 & 0.39 & 0.71 \\ 
  8 & 0.96 & 0.97 & 0.52 & 0.97 & 0.22 & 0.61 \\ 
  9 & 0.89 & 0.97 & 0.89 & 0.98 & 0.39 & 0.62 \\ 
  10 & 0.98 & 0.78 & 0.22 & 0.98 & 0.18 & 0.47 \\ 
  11 & 0.97 & 0.93 & 0.80 & 0.97 & 0.24 & 0.81 \\ 
   \hline
\end{tabular}
\end{table}

This is probably mostly a sample size problem.

\section{Poisson Regression Works Like a Charm}

The p-value table for the ethnicities too.

% latex table generated in R 4.0.3 by xtable 1.8-4 package
% Mon May 24 03:17:15 2021
\begin{table}[ht]
\centering
\begin{tabular}{rrrrrrr}
  \hline
 & 1 & 2 & 3 & 4 & 5 & 6 \\ 
  \hline
1 & 74989.72 & 0.00 & 0.00 & 1.77 & 0.00 & 0.00 \\ 
  2 & 2211914.24 & 0.00 & 0.00 & 0.00 & 0.00 & 0.00 \\ 
  3 & 67886944.45 & 0.00 & 0.00 & 0.05 & 0.00 & 0.00 \\ 
  4 & 0.14 & 0.00 & 0.00 & 0.00 & 0.00 & 0.00 \\ 
  5 & 218186.44 & 0.00 & 0.00 & 0.00 & 0.00 & 0.00 \\ 
  6 & 0.00 & 0.00 & 0.00 & 0.00 & 0.00 & 0.00 \\ 
  7 & 28453454.37 & 0.00 & 0.00 & 0.00 & 0.00 & 0.00 \\ 
  8 & 1758720.25 & 0.00 & 0.00 & 0.00 & 0.00 & 0.00 \\ 
  9 & 187725673.21 & 0.00 & 0.00 & 131370152.16 & 0.00 & 0.00 \\ 
  10 & 0.00 & 0.00 & 0.00 & 27924043.55 & 0.00 & 0.00 \\ 
  11 & 0.00 & 0.00 & 0.00 & 427171520.88 & 0.00 & 0.00 \\ 
   \hline
\end{tabular}
\end{table}

These are p-values multiplied by $10^{24}$ they were so small when Poisson regression was fit to ethnicities too.

The variation due to ethnicities are here.

\begin{verbatim}
> for (r in 1:11){ a<-abs(A[r,]);pv<-(1-exp(-a*5))/(1-exp(-a*10));print(sd(pv))}
[1] 0.09423666
[1] 0.1250242
[1] 0.1467352
[1] 0.05393183
[1] 0.04052609
[1] 0.05888644
[1] 0.08703167
[1] 0.09612605
[1] 0.1410823
[1] 0.09550833
[1] 0.1084014
\end{verbatim}

The mean of these variations are 
\[
\bar{\sigma}_{eth} = 0.0952
\]

So {\em this is the variation of child rearing variation due to ethnicity}! Excuse me, but {\em HAHAHAHAHAHAHAHAHA

HAHAHAHAHAHAHAHAHA

HAHAHAHAHAHAHAHAHA
}


All the racial theorists will have a pretty difficult time convincing anyone of their rather absurd theories of racial superiority with only 9.52\% at play.

\section{Code}

\begin{verbatim}
# We will use logistic regression
# with x variable artificially created for Q7-Q17
# We take N=500 points to determine p, lambda
# Then we use these to assign random x values
# for all the other values
# then we fit logistic regression on the (x,g)

samp.binary.exp<-function( grps, lambda ){
  grps<-as.vector(t(grps))
  n <- length(grps)
  print(n)
  gvals <- unique(grps)
  #print(gvals)
  bval <- 0
  sval <- 1
  if ( length(gvals) == 2 ){
    if ( sum(gvals==gvals[1]) >= n/2 ){
      bval <- gvals[1]
      sval <- gvals[2]
    } else {
      sval <- gvals[1]
      bval <- gvals[2]
    }
  } else {
    return(NULL)
  }
  xs<- rep( 0, n)
  for ( r in 1:n ){
    done <- F
    
    while ( done == F){
      pickx <- rexp(1,rate=1/lambda)
      #print(pickx)
      if (pickx <= log(2)/lambda){
        if ( grps[r] == sval){
          xs[r] <- pickx
          done <- T
        }  
      }
      if (pickx > log(2)/lambda){
        if ( grps[r] == bval){
          xs[r] <- pickx
          done <- T
        }  
      }
    }
  }
  xs
}

dataset.logit<-function(var,lambda){
  y<-na.omit(polv[,var])
  G<-as.numeric(as_factor(t(y)))-1
  p<-sum(G==0)/length(G)
  if ( p < 0.5) {
    G<-1-G
  }
  p<-sum(G==0)/length(G)
  print(p)
  
  x<-samp.binary.exp( G, lambda)
  out<-data.frame( x=x,G=G)
  names(out) <- c("x","G")
  out
}

dataset.logit.fixed.lambda<-function(var){
  y<-na.omit(polv[,var])
  G<-as.numeric(as_factor(t(y)))-1
  p<-sum(G==0)/length(G)
  if ( p < 0.5) {
    G<-1-G
  }
  p<-sum(G==0)/length(G)
  print(p)
  lambda <- 2*log( p/(1-p) )
  #print(head(y))
  x<-samp.binary.exp( y, max(lambda,2.4))
  out<-data.frame( x=x,G=G)
  names(out) <- c("x","G")
  out
}

if (FALSE){
lq14<-dataset.logit.fixed.lambda("Q14")
mq14 = glm( G ~ x, family="binomial",data=lq14)
summary(mq14)
}



dataset.logit.eth<-function(var,eth){
  y0<-na.omit(polv[,var])
  G0<-as.numeric(as_factor(t(y0)))-1
  p0<-sum(G0==0)/length(G0)
  if ( p0 < 0.5) {
    G0<-1-G0
  }
  p0<-sum(G0==0)/length(G0)
  lambda0 <- 2*log( p0/(1-p0) )
  #lambda <- max(min(lambda,2.7),1.5)
  lambda0 <- max(lambda0,2.2)
  #print(head(y))
  x0<-samp.binary.exp( y0, lambda0)
  
  idx<-as.character(polv$eth)==eth
  y<-na.omit(polv[idx,var])
  G<-as.numeric(as_factor(t(y)))-1
  x<-x0[idx]
  p<-sum(G==0)/length(G)
  if ( p < 0.5) {
    G<-1-G
  }
  p<-sum(G==0)/length(G)
  #print(p)
  lambda <- 2*log( p/(1-p) )
  out<-data.frame( x=x,G=G)
  names(out) <- c("x","G")
  list(df=out,lambda=lambda)
}

dataset.logit.eth2<-function(var,eth){
  y<-na.omit(polv[as.character(polv$eth)==eth,var])
  G<-as.numeric(as_factor(t(y)))-1
  p<-sum(G==0)/length(G)
  if ( p < 0.5) {
    G<-1-G
  }
  p<-sum(G==0)/length(G)
  print(p)
  lambda <- 2*log( p/(1-p) )
  #lambda <- max(min(lambda,2.7),1.5)
  lambda <- min(max(lambda,0.5),2)
  #print(head(y))
  x<-samp.binary.exp( y, lambda)
  out<-data.frame( x=x,G=G)
  names(out) <- c("x","G")
  list(df=out,lambda=lambda)
}

vars<-c("Q7","Q8","Q9","Q10","Q11","Q12",
        "Q13","Q14","Q15","Q16","Q17")

eths<-c("Arab", "Black","East Asian", "Indian", "Other", "White")

child.eth<-function(){
  lambdas <- matrix( 0, nrow=length(vars), ncol=length(eths))
  alphas <- matrix( 0, nrow=length(vars), ncol=length(eths))
  pvals <- matrix( 0, nrow=length(vars), ncol=length(eths))
  bdf <- data.frame()
  for (r in 1:length(vars)){
    for (s in 1:length(eths)){
      A<-dataset.logit.eth( vars[r], eths[s])
      m <- glm( G ~ x, family="poisson", data=A$df)
      lambdas[r,s]<-A$lambda
      alphas[r,s]<-summary(m)$coefficients[2,1]
      pvals[r,s]<-summary(m)$coefficients[2,4]
      bdf<-rbind(bdf,c(vars[r],eths[s],alphas[r,s],lambdas[r,s],pvals[r,s]))
    }
  } 
  names(bdf)<-c("var","eth","alpha","lambda","pval")
  list(df=bdf,lambda=lambdas,alpha=alphas,pvals=pvals)
}
\end{verbatim}
\end{document}