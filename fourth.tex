\documentclass{amsart} 
\usepackage{graphicx}
\graphicspath{{./}}
\usepackage[fontsize=14pt]{scrextend}
\usepackage{hyperref}
\title{Rilke's Fourth Elegy}
\author{Zulfikar Moinuddin Ahmed}
\date{\today}
\begin{document}
\maketitle

\section{You Don't Want to Live This Way}

You were not born on this planet, a living human being to go through your schooling and become a corporate drone working on this and that project so other people will make money off your work so that you can exploit some others.  You know that that is unsatisfactory.  So you make your peace; you follow your passions, and curse once in a while under your breath and keep your focus on your family and your lover or wife, your children, your friends and manage your sense of meaningful life.  And is that satisfactory to you?  You say it could be worse.  Well Rilke would have committed suicide long before this.  Rilke's demands were vastly more than yours -- if you are of ordinary thick skin and resilience.  "These thin-skinned artists , God bless them".  Well let me be the Acolyte of Rilke for you.  Be much more unsatisfied my friends, your satisfaction level is nauseating.  You will not find me kneeling in front of your statue and weeping in self-deprecation is you are satisfied.  Seek past your death and be dissatisfied with your inauthenticity, don't be shy, you'll live.  You can't even believe what you see, can you?  You can't live like that.  

\section{Transcendental Perspective of Death}

I return to the Fourth Elegy of Rilke.  It is a particular resolution of a transcendent perspective of our lives from the perspective of death by the idea that we are born with our death growing within us that we hold gently and not refuse to go on living.

The poem begins with our disharmony with nature and our weakness: we are not as able to tell the coming winter through our blood and we are not able to be as powerful as the lion. The contours of our emotions are unknown to us. Behind our heart's curtain we see ourselves in a farewell -- which must have seemed so real then -- seem as a half-filled mask of a dancer.  Dead can see that even our most earnest emotions are full of pretense and unreality, not filled with Beingness say of the Angel.  The poem is an a searching gaze across our lives from the perspective of death who are more able to detect reality, and whence cometh this falseness of our lives?  How to resolve this existential dissatisfaction?  We are born with death within us, which is not negative but essential to what gives us reality that we hold gently without refusing to go on.

\end{document}

