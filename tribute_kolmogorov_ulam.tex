\documentclass{amsart}
\usepackage{graphicx}
\graphicspath{{./}}
\usepackage{hyperref}
\usepackage{csvsimple}
\usepackage{longtable}
\usepackage{lscape}
\usepackage{epigraph}
\title{A Tribute To A. N. Kolmogorov and S. Ulam and Paul Levy}
\author{Zulfikar Moinuddin Ahmed}
\date{\today}
\begin{document}
\maketitle

\section{Brief History}

I am reading Richard Dudley's graduate Analysis and Probability book to refresh basic material.  His historical notes are quite good.  S. Ulam actually defined probability axioms in 1932, then there was Kolmogorov 1933, then Lomnicki-Ulam mention the Ulam 1932 and Kolmogorov 1933 in a paper of 1934.  For two decades before probability theory was only for Euclidean spaces and [0,1].

This history makes it clear just how important A. N. Kolmogorov had been as well as S. Ulam, because I am literally fitting infinitely divisible laws that are precise (Barndorff-Nielsen Laws) to actual data from Social Sciences today, let me see, 88 years later.  Without Kolmogorov and Ulam setting up the probability theory, there would have been very little conviction for even attempting to fit infinitely divisible precise laws to ordered categorical value data.  

I am not the envious type; I appreciate the great gifts of Ulam and Kolmogorov and Paul Levy that allowed Zulf to see further than I could have even dreamt of seeing.  Truly fabulous things they had done.  And I am glad they did it, because I am quite a bit more devoted to Nature's secrets and would not have the interest to set up wide foundations with such beautiful care.  I have a copy of A. N. Kolmogorov's book.  The abstract theory was necessary to make any sense of precise processes like Barndorff-Nielsen.  We are fortunate to have such great geniuses in our past.  

What would Zulf do if they never laid the foundations of probability theory?  I would be lost, and Nature's secrets she would not share with Zulf.

\section{Why I Do Not Envy A. N. Kolmogorov At All}

A. N. Kolmogorov was an extraordinary genius.  He worked on turbulence models.  He proved some of the most important foundational theorems in probability including the basic Strong Law of Large Numbers.  He constructed Fractional Brownian Motion long before Benoit Mandelbrot used it to great effects.  

But I do not envy a man like this.  I admire him, but I have no envy at all for him.  It is because genius is only possible with love, all genius.  And he loved something about probability in this massive manner, a bit like David Hilbert.  And his love and passion for these abstract theories that are concretised with theorems that establish foundations and stretch space of intellectual grandeur, I do not have the same passion for those things.  It's not a frivolous thing.  I love other things, and I am glad that A. N. Kolmogorov did the things he did and made the world more enlightened.  It is impossible to do anything of value without deep love.  I do not have problems with what I love.  And I stand on things that A. N. Kolmogorov had provided from their own.

\section{Model of Uncountably Infinite Human Beings}

There are 7.8 billion humans on Earth.  Let $H_0$ be a finite set representing them with $|H_0| = 7.8 \times 10^{9}$.  We are not satisfied with this in our view of the world.  We will consider embedding $H_0$ in a Euclidean space $H_0 \subset H$.  This extension is valuable for us.  This is crucial for us.  

We want to take continuous densities seriously on human race measurements.  If we leave finite $H_0$ then if we consider the measurement $v: H_0 \rightarrow \mathbf{R}$ then we cannot have a smooth density for the distribution of $v$.  This is easy to see because the $v(H_0)$ is finite, and therefore we can find an interval in the complement of $v(H_0)$ and the distribution of $v$ cannot have positive mass on it.  

We will extend human race to be represented by $H$ which is a continuous parameter space, say $H \subset \mathbf{R}^G$.  This will allow us to consider functions with distributions that have a smooth density on $\mathbf{R}$.




\end{document}