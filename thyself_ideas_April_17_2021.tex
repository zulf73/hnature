\documentclass{amsart} 
\usepackage{amsmath}
\usepackage{verse}
\DeclareMathOperator*{\argmax}{arg\,max}
\DeclareMathOperator*{\argmin}{arg\,min}
\usepackage{graphicx}
\graphicspath{{./}}
\usepackage[fontsize=14pt]{scrextend}
\usepackage{hyperref}
\usepackage{csvsimple}
\usepackage{epigraph}
\title{A New View Toward Uniqueness in the Context of Quantitative Universal Human Nature}
\author{Zulfikar Moinuddin Ahmed}
\date{\today}
\begin{document}
\maketitle

\section{Basic Features of Universal Human Nature}

Our efforts at establishing a Quantitative Human Nature involves understanding the Analytic Infrastructure that will one day yield a scientific understanding of Human Nature that is accurate.  We have found three features that are concrete and revolutionary in Social Science.  

\begin{itemize}
\item Ubiquity of Generalised Hyperbolic Distributions
\item Invariant Empirical Distributions to sampling from many countries pooled together
\item Exponential Laws for Moral Values
\end{itemize}
These new discoveries of ours gives us a rough picture of Human Nature as finite dimensional continous system of distributions.  In other words even though for the human race $\mathcal{H}$ effectively $|\mathcal{H}| = \infty$ we expect a much lower finite dimensional total model of Human Nature in the future.

This is the context of understanding Uniqueness of individuals.  We will expect two dimensions $D_m << D_u$ where $D_m$ is the dimension of a Human Nature Model and $D_u$ is the dimension of another continuous model that is able to account for uniqueness of individuals.  The idea of Thyself Inc. was originally based on Personality Psychology and Technology marriage, and now we can see the Scientific Context.  

These are extremely general considerations, but I find it useful to consider the theoretical foundations of Uniqueness. 

\section{Careful Merging of Points of View}

I envisioned Thyself Inc. originally as a company to provide goods and services to enhance life for all human beings, very general idea.  It is very important to focus attention on the difficulties of handling 7.8 billion people directly rather than rushing to handle 100,000 people and then expand outward.  This is because this larger point of view allows us to see how many algorithmic approaches that are reasonable for 100,000 people simply would not serve in a 7.8 billion environment.  The Technology concept of {\em scalability} itself is the wrong concept.  That concept assumes scaling of smaller scale problems.  Our viewpoint is implementation of uniqueness models that are fundamentally continuous in nature.  We believe that Uniqueness of individuals are best served by continous models because we unnecessarily welcome combinatorial complexity without Analytic handle on the situation.

Slowly we find Analytical models that are accurate, and based on these design systems where billions can be handled with more analytic tools.  Unfortunately I will not have examples tonight.  

We remind the reader that models of psychological types of individuals are worthless unless they are approximately accurate.  So we need to get to the point where our conception of the individual is not ad hoc but unified with Universal Human Nature.

These seemingly abstract considerations in my view is actually the least painful method of having both accurate handling of individuals that is accurate and tractable.  

\section{Cultivation, Taste, and High Culture Necessary}

In order to design systems that 7.8 billion find appealing, one cannot rely on engineering creativity and cleverness but must also rely on the actual evolution of Human Civilisation in the past six thousand years.  Human beings want to be uplifted by the highest cultures they have access to.  Democratic assumptions have some mileage, but in the end people want high culture and civilisation.  This means taste is paramount as well as cultivation for success of a 7.8 billion show for goods and services.  The engineering approaches of massive scalability worked for Amazon but will not work for products to improve human life because people use technology as a tool to further other goals in life, and those goals are addressed best by high culture rather than the democratic culture of convenience.

Here I am thinking about the limitations of Microsoft.  Microsoft has always rushed to scale and sell first and keep fixing and updating products.  This produced a gigantic amount of work just coordinating systems within Microsoft and gigantic "Knowledge Databases" to handle the vast complexities that are mostly internal to Microsoft.  This is the engineering approach at its limits.  The alternative approach is Scientific Analytic.  In this approach we consider the 7.8 billion management directly and attempt to understand individuals in the context.  The analogy of smooth function analysis to discrete mathematics will give sense of the problem with the engineering approach.  The engineering approach is extremely effective in smaller scales but is totally blind in the largest scales.  In the largest scales combinatorial methods will collapse and produce no uniqueness services worth much because it's too costly to manage without Analytic understanding.  I am not sure that I am expressing the issue properly here.  To the extent that uniqueness of seven billion can be handled at all, this cannot be done with any level of seriousness without continuous models.  Even a 200,000 staff company will find the problem of service to 7.8 billion {\em on uniqueness} completely overwhelming because discrete computer science thinking without a continuous science is fundamentally unable to handle 7.8 billion.  The principles of handling 7.8 billion have not been formulated to anyone's satisfaction yet.


\end{document}