\documentclass{amsart}
\usepackage{graphicx}
\graphicspath{{./}}
\usepackage{hyperref}
\usepackage{csvsimple}
\usepackage{longtable}
\usepackage{lscape}
\usepackage{epigraph}
\title{Mystical Elements in Zulf's Philosophy of Science} 
\author{Zulfikar Moinuddin Ahmed}
\date{\today}
\begin{document}
\maketitle

\section{Mysticism of a Sort is Compatible with Science}

I am not a three-dimensional empiricist.  I am extremely talented as a Scientist, but I am not an Empiricist in any classical way.  I want to share with you my Scientific Philosophy a bit better.  Science between 1650-1800 say, developed an Empiricist orientation.  I appreciate the empirical orientation, but this is not actually what I believe is right.  I was ever a Platonist when I was younger, and as I grow older, I have a more mystical view of Science.

My particular {\em mysticism} is not exactly the religious mysticism of Sufi Muslims or Christians in rapture or Hindu mendicants and so on.  It is a mysticism of a different sort.  It is the mysticism that continuously seeks to understand Absolute Truth and the Limits of Human Mind to capture it.  Now to me Four-Sphere Theory is Absolute Truth; it is not the 'Whole Truth' but it is Absolute Truth.  It is, for me at least, no ordinary scientific theory.  It's just Truth.  So Eternal Recurrence, non-Expansion of the Universe, a vast large purely electromagnetic spatial dimension in the actual universe to me are just {\em facts}.  For me, there is no possibility of any mere physics {\em theory} ever competing with it.  It would be like me taking a swim in a Zen pond with fishies of all colours every day and dealing with Theologians in Brooklyn having deep discussions about the impossibility or possibility of existence of Zen ponds and colourful fishies.  For me, Four-Sphere Theory is concrete Absolute Truth.

But even though Four-Sphere Theory is in principle a complete mathematical physics down to $10^{-13}$ cm, in fact, this is quite poor knowledge of the richness of the universe.  

\section{The Background of My Scientific Mysticism}

There is a vast purely electromagnetic spatial dimension in the universe whose range is 3075.69 Mpc and that is not experienced in the same way as any part of previous science had even imagined.  This alone makes objective Nature quite a bit less 'experiential' than the three-dimensional constraints.  S4 Electromagnetism operates on the whole of the ambient universe, a four-sphere.  So Absolute Truth contains concrete extra large purely electromagnetic dimension.  This dimension is not speculative and exists in fact.  

Now this does not invalidate any Science that produces good predictions on repeated feature.

Now there is a serious question about how we will make Science {\em coherent} with this?  Well, the physical universe is an evolving hypersurface and the S4 EM is the only macroscopic law.  So in principle things ought to be deductions.  But they never are because Nature is filled with actual mysteries that we will not decipher without new insights and discoveries.  

\section{Paul Feyerabend's Classical Empiricism Essay}

Paul K. Feyerabend in "Classical Empiricism" does some marvelous analysis of Baconian empiricist philosophy based on 'experience' as the source of all knowledge.  I am firmly opposed to experience as the source of all knowledge, and therefore am a Mystic.  By 'experience' of course I understand the experience of the physical world, the five senses, additional instruments that observe and such.  Instead I am and always will be a Platonist and a Mystic.  Now seventeenth century and twenty first century are different in that Empiricism has gone through the whole loop and in 2018 I could declare Four-Sphere Theory as a coherent Mathematical Physics that will cover all macroscopic (low energy, down to $10^{-13}$ cm) laws.  And a great deal of empiricism went into establishing the basis for my success.

Then I can be more Platonic and Mystical {\em again} and place Empiricism as a useful tool and devote myself to understanding the deeper truth behind appearances {\em with a firm mathematical physics} that I have faith is Absolute Truth.  And so Mystical-Platonic fundamentalism is available to me because the Empiricist extravaganza has gone through their success.  And I have no shame whatsoever, in grabbing hold of the fruits of three centuries of Empiricism, putting all of it in a coherent Mathematical Physics and waving my hands dismissively as these {\em plebian riffraff, these democratic fools who only think that that which can be seen with the eyes and felt with the hands, these museless wretches}.  

\section{What Is Science}

Science is knowledge regarding Existence.  It is knowledge regarding the World and all things of Heaven and Earth.  I am not an Empiricist at all.  I think of the World in which I live, Existence, as quite vast and complex, and the aim of Science is to provide us with {\em certain knowledge} about all of it.  That is the grand ambition of Science.  It is unlimited ambition.  Now I happen to consider Nature to be quite a bit more elaborate than the world of five senses and three dimensions.  For me there is no problem with purely electromagnetic life; such things may or may not exist by heavy measurements, but they would not be {\em supernatural} for me.  Why?  I live in the Universe described by Four-Sphere Theory.  I don't live in the primitive world of Expansionary Cosmology and Relativity Theory and Schroedinger Quantum Mechanics, and three-dimensions or bust. So possible phenomena are wider, and I know the Mathematical Physics for all of macroscopic universe including unseen parts of the fourth spatial dimensions.  

I see Empiricism as a tool that Science requires, but I do not see it fruitful to demand that Empiricism is the only way forward.  At the same time, sharp and careful inference from measurements is important, it's necessary.  It is necessary to get involved in some empirical work.  Scientific models require this, but that does not imply that we ought to turn Empiricism into our guiding philosophy.  Just because I am Platonic Mystic does not mean that I am not just as competent as any hard-core Empiricist at fitting models to data.  It's necessary to get closer to the Secrets of Nature, but to demand that Empiricist philosophy will lead to those secrets is absurd.  No one knows what will elicit all of Nature's Secrets at all.  Given this, it is the height of folly to restrict oneself to the rough discipline of the Empiricist, the drone worker's philosophy that will suck all the joy and pleasures of life from my short life, and expect any sense.  One must treat one's subjective reality and one's happiness and comfort fairly in order that motivation still exists tomorrow to seek the Divine Truths of Nature.  Dogmatism of low brow Empiricism will not guarantee this, and I cannot stand the accountants that populate the ranks of hard-nosed empiricists with their obnoxious lack of education in every other aspect of civilisation.  They should mark themselves with pocket protecters and slide rules so I can conveniently disappear when they arrive in any social gathering.

\section{No Oracle of Nature Exists}

It has been a great dream of human psychological satisfaction that profound truths of Nature will be revealed by Tiresias the Blind Prophet of Apollo, whose blindness suggests an inner eye; the Oracle of Delphi was the source of divine truth.  With Science, there is no such Oracle, and we are left to our own devices.  

My Mystical Faith is that Human Race is capable of understanding some laws of Nature at a nontrivial level, but this is challenging quest, a challenging devotion, the pursuit of Truth, for Nature has no obligation to ever respect our concepts, and forever we are in the situation that our rules for ensuring that we are actually discovering firmer ground of truth will fail at some level, and we must review our understanding.  

I know this all too well, for it was I who had overthrown gravity itself with S4 Electromagnetism, the subject of Isaac Newton's Universal Law of Gravitation.  Here was a case par excellence of a great scientific mind not being able to penetrate sufficiently the depths of nature.  My refutation with Van der Waals absorbing gravity into electromagnetism was relatively simple in 2010s but it would have been impossible in 1700s, because I could just find the nature of Van Der Waals measured from basic physical chemistry texts.  Isaac Newton was the pioneer of the whole enterprise.  In fact, Four-Sphere Theory is perfect; gravity does not exist.  I polished an eternal clear model of Nature and Nature is actually just governed by the elements of Four-Sphere Theory.  I know this with more certainty than I know the lines on the palm of my hand.

The inscrutibility of Nature is not strange; and seeking her secrets is not strange either.  Mystics like me are Reverent and respectful towards the inscrutibility, even as we gain more actual truth we are constantly aware that we did not create Nature and her laws, so we ought to be extremely sensitive to attempting to foist on her something that is not hers but ours.  

Four-Sphere theory is truth.  If I had to create Nature by my own ways, I would not use a Four-Sphere.  But Four-Sphere is what is, is Nature, and far be it from me to question what is Nature.


\end{document}

