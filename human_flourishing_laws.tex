\documentclass{amsart} 
\usepackage{amsmath}
\DeclareMathOperator*{\argmax}{arg\,max}
\DeclareMathOperator*{\argmin}{arg\,min}
\usepackage{graphicx}
\graphicspath{{./}}
\usepackage[fontsize=14pt]{scrextend}
\usepackage{hyperref}
\usepackage{csvsimple}
\usepackage{epigraph}
\title{Search for Human Flourishing Laws}
\author{Zulfikar Moinuddin Ahmed}
\date{\today}

\begin{document}
\maketitle

\section{Personal Considerations}

When I was young, I was deeply moved by Mathematics and literature of Dostoevsky and Kafka, and I immediately took to Nietzsche during my teenage years.  I was not particularly interested in Science in the hands-on sense.  I was not an Empiricist but a Platonist.  And for these reasons, it took me a very long time to truly appreciate the achievement of Charles Darwin, the Naturalist. For me, there is nothing to really doubt about the theory of Evolution itself, for the theory has a great deal of physical evidence.  

What bothers me still, however, is that no Quantitative Science of Human Nature had developed that addressed questions of optimal understanding that can assist seven billion people raise children for a flourishing Human Civilisation.  

This is not a theoretical frivolous exercise, not just academic, but a serious issue.  Since we did not have certain or Scientific understanding of Human Nature that applies of seven billion people, we have not sought universal understanding of Human Psychological Health, the process of raising children, the social processes and modes of behaviour that will lead to a flourishing World Civilisation.  We are still held back by divisions and strife across the planet.  

\section{Universal Laws for Flourishing Exist}

The results I have found in the past few years and months on issues of universal Human Nature make it quite clear than there are many issues where we can have clear quantitative understanding that could assist vast numbers of human beings change their ways slightly to gain enormous benefit for themselves and the future of their children.  These are the sorts of Universal Laws that I think are valuable.  

\section{Looking Beyond Divisions of Race, Religion, Ethnicity, Culture}

We live at the end of around five centuries of colonialism across the globe, after a couple of world wars, and then a global ideological struggle.  We know quite well that we cannot artificially induce flourishing, that the organic developments of culture are not frivolous, that traditions have also gone through adaptations that we do not understand fully yet.  At the same time, we can take a light approach and consider Universal Human Nature understanding and make small recommendations that are beneficial for a better future for all people everywhere.  We have no choice in the matter, for either we will have a good Science, or we will leave the future of our people to very bad ideas such as racial domination of the planet and so on.  

\section{New Science of Quantitative Universal Human Nature is Necessary}

Just as Darwin's ideas were fruitful and difficult, we are at a point in Human History where we will risk great suffering if we do not take active initiative to work out a Quantitative Science of Human Nature which is universally applicable.  This requires a lot of effort, and Aesthetics is key to it.  We, the Human Race, have always been an Aesthetic species.  We can see that in the way we had developed our myths in Mesopotamia, in Egypt, in China, in Greece and everywhere.  We use Beauty to know what to preserve and what to forget.  And so the Search For a Quantitative Science of Human Nature will challenge our Aesthetic Abilities, and yet it is a profound Duty.

\end{document}