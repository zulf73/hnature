\documentclass{amsart}
\usepackage{graphicx}
\graphicspath{{./}}
\usepackage{hyperref}
\usepackage{csvsimple}
\usepackage{longtable}
\usepackage{epigraph}
\title{Fresnel Integral Computation}
\author{Zulfikar Moinuddin Ahmed}
\date{\today}
\begin{document}
\maketitle

This is not research but a complex analysis exercise from Stein-Shakarchi Complex Analysis.

Compute
\[
\int_0^\infty \sin(x^2) dx
\]

\section{Major Steps}

The key step is to note that rotation by fourth root of unity $\theta_4 = e^{i\pi/4}$ allows us to consider the following scenario.

On the slanted line we have $e^{-t^2}$.  On the horizontal line we have $\sin(x^2)+i \cos(x^2)$.

This is the crucial step because the slanted line integral is known.  It is
\[
\int_0^{\infty} e^{-x^2} dx = \sqrt{\pi}/2
\]


We need to take limit as $R\rightarrow\infty$ and integrate around a closed contour.  The arc will have
\[
| \int_0^{\pi/4} e^{-R^2 g(\theta)} d\theta| \le C e^{-R^2}
\]
So it will vanish in the limit.

The algebra for exact computation is
\[
\frac{\sqrt{\pi}}{2} = | a + i a|
\]
This then leads to
\[
2a^2 = \pi/4
\]
This gives $\sqrt{\pi/8} = \sqrt{2\pi/16} = \sqrt{2 \pi}/4$.  

\section{Beauty of Mathematics Here}

The use of $\int_{-\infty}^\infty e^{-x^2} dx = \sqrt{\pi}$ by rotation in the complex plane is just remarkable.  This is an eighteenth century affair with complex plane entering calculus.  It was Augustin-Louis Cauchy who wanted a rigorous theory of functions to explain how these miraculous change of variables produced correct answers, which developed into the standard vanishing of the arc integral.  Here it is a simple estimate. The last part in algebra.


\end{document}