\documentclass{amsart} 
\usepackage{amsmath}
\DeclareMathOperator*{\argmax}{arg\,max}
\DeclareMathOperator*{\argmin}{arg\,min}
\usepackage{graphicx}
\graphicspath{{./}}
\usepackage[fontsize=14pt]{scrextend}
\usepackage{hyperref}
\usepackage{csvsimple}
\usepackage{epigraph}
\title{Basic Examination of Honesty in Human Nature}
\author{Zulfikar Moinuddin Ahmed}
\date{\today}

\begin{document}
\maketitle

\section{Who I Am}
At 48, I am still following the American Dream, with a registered company,  Thyself Inc., still trying to get my money, around \$120 million, owd to me by D. E. Shaw \& Co. being blockaded by Bill Gates.  Unlike many other people, for me defining moment in my life was university, and I graduated Princeton University in 1995, magna cum laude.  Then I came on board the group of Andrew Morton at Lehman Brothers Fixed Income Derivative Research.  I am given to nostalgic reminisces, so these are parts of my history hallowed and all blemishes have been washed away by time.  I am responsible for the Four-Sphere Theory which is right and is a Scientific Revolution in itself, and this note continues by examination in Social Science.

\section{Our Null Model For Human Virtues}

Let us take a step back to examine our view of Human Race.  This is a species that has evolved for six million year in Africa, and all non-African populations diffused from East Africa 75,000 years ago.  This is as close to a rough model that is absolute truth as we can find today, and certainly strong scientific consensus.  Our view is that Virtues and Morals developed in that six million years in some {\em equilibrium state} before our diffusion, and then there have been relatively minute fluctuations.  This is our null model.  

In other words, we believe that morals and virtues, since they are functionals of our social instincts, which developed 200 million years ago, are roughly the same across the globe because even though individual virtues and morals can have wild variations, their basis is deep in our genetic history, their mechanisms are encoded in genes, and for the global populations the mean levels have an impossibly difficult time changing.  The means have changed very slowly over hundreds of thousands of years.  Here we want to be precise.  Morals of populations change over thousands of years and not millions.  Thus when we say 'thousands' or 'millions' these are not rhetorical flourishes but require being considered quantitatively.  Two hundred million years is the evolution of emotions in mammals, roughly.  The social instincts of human beings are six million years old.  The finer structures of morals and virtues of human beings do change in millenia.  But large populations have difficulties changing in a few decades.  Sometimes forces do push forward and strain deep genetic adaptations of humans; those then produce vast numbers of problems are in fact Darwinian pressures.  

The point is not simply to take 'evolutionary viewpoint'.  Evolution is true, so it's just one background truth.  We are interested in understanding Human Nature, and do not stand only on evolutionary explanations.


\end{document}